%%%%%%%%%%%%%%%%%%%%%%%%%%%%%%%%%%%%%%%%%
% Masters/Doctoral Thesis 
% LaTeX Template
% Version 2.5 (27/8/17)
%
% This template was downloaded from:
% http://www.LaTeXTemplates.com
%
% Version 2.x major modifications by:
% Vel (vel@latextemplates.com)
%
% This template is based on a template by:
% Steve Gunn (http://users.ecs.soton.ac.uk/srg/softwaretools/document/templates/)
% Sunil Patel (http://www.sunilpatel.co.uk/thesis-template/)
%
% Template license:
% CC BY-NC-SA 3.0 (http://creativecommons.org/licenses/by-nc-sa/3.0/)
%
%%%%%%%%%%%%%%%%%%%%%%%%%%%%%%%%%%%%%%%%%

%----------------------------------------------------------------------------------------
%	PACKAGES AND OTHER DOCUMENT CONFIGURATIONS
%----------------------------------------------------------------------------------------

\documentclass[
11pt, % The default document font size, options: 10pt, 11pt, 12pt
oneside, % Two side (alternating margins) for binding by default, uncomment to switch to one side
english, % ngerman for German
singlespacing, % Single line spacing, alternatives: onehalfspacing or doublespacing
%draft, % Uncomment to enable draft mode (no pictures, no links, overfull hboxes indicated)
%nolistspacing, % If the document is onehalfspacing or doublespacing, uncomment this to set spacing in lists to single
%liststotoc, % Uncomment to add the list of figures/tables/etc to the table of contents
%toctotoc, % Uncomment to add the main table of contents to the table of contents
%parskip, % Uncomment to add space between paragraphs
%nohyperref, % Uncomment to not load the hyperref package
headsepline, % Uncomment to get a line under the header
%chapterinoneline, % Uncomment to place the chapter title next to the number on one line
%consistentlayout, % Uncomment to change the layout of the declaration, abstract and acknowledgements pages to match the default layout
table,
]{MastersDoctoralThesis} % The class file specifying the document structure

\usepackage[utf8]{inputenc} % Required for inputting international characters
\usepackage[T1]{fontenc} % Output font encoding for international characters

\usepackage{mathpazo} % Use the Palatino font by default
\usepackage{float}

\usepackage[backend=bibtex,style=authoryear,natbib=true,doi=false]{biblatex} % Use the bibtex backend with the authoryear citation style (which resembles APA)

\addbibresource{references.bib} % The filename of the bibliography

\usepackage[autostyle=true]{csquotes} % Required to generate language-dependent quotes in the bibliography
\usepackage{eurosym}

\usepackage{amsmath,amsfonts,amssymb,amsthm}
\usepackage[linesnumbered,ruled]{algorithm2e}
\SetKwInput{KwInput}{Input}
\SetKwInput{KwOutput}{Output} 

\usepackage{subcaption}

%----------------------------------------------------------------------------------------
%	MARGIN SETTINGS
%----------------------------------------------------------------------------------------

\geometry{
	paper=a4paper, % Change to letterpaper for US letter
	inner=2.5cm, % Inner margin
	outer=3.8cm, % Outer margin
	bindingoffset=.5cm, % Binding offset
	top=1.5cm, % Top margin
	bottom=1.5cm, % Bottom margin
	%showframe, % Uncomment to show how the type block is set on the page
}

%----------------------------------------------------------------------------------------
%	THESIS INFORMATION
%----------------------------------------------------------------------------------------

\thesistitle{Analysis of the SVM-RFE algorithm for feature selection} % Your thesis title, this is used in the title and abstract, print it elsewhere with \ttitle
\supervisor{Luis A. \textsc{Belanche}} % Your supervisor's name, this is used in the title page, print it elsewhere with \supname
\examiner{} % Your examiner's name, this is not currently used anywhere in the template, print it elsewhere with \examname
\degree{Bachelor Degree in Informatics Engineering} % Your degree name, this is used in the title page and abstract, print it elsewhere with \degreename
\author{Robert \textsc{Planas}} % Your name, this is used in the title page and abstract, print it elsewhere with \authorname
\addresses{} % Your address, this is not currently used anywhere in the template, print it elsewhere with \addressname

\subject{Computing} % Your subject area, this is not currently used anywhere in the template, print it elsewhere with \subjectname
\keywords{} % Keywords for your thesis, this is not currently used anywhere in the template, print it elsewhere with \keywordnames
\university{\href{http://www.upc.edu}{Universitat Politècnica de Catalunya}} % Your university's name and URL, this is used in the title page and abstract, print it elsewhere with \univname
\department{ \_ } % Your department's name and URL, this is used in the title page and abstract, print it elsewhere with \deptname
\group{ \_ } % Your research group's name and URL, this is used in the title page, print it elsewhere with \groupname
\faculty{Facultat d'informàtica de Barcelona} % Your faculty's name and URL, this is used in the title page and abstract, print it elsewhere with \facname

\AtBeginDocument{
\hypersetup{pdftitle=\ttitle} % Set the PDF's title to your title
\hypersetup{pdfauthor=\authorname} % Set the PDF's author to your name
\hypersetup{pdfkeywords=\keywordnames} % Set the PDF's keywords to your keywords
}

\begin{document}

\frontmatter % Use roman page numbering style (i, ii, iii, iv...) for the pre-content pages

\pagestyle{plain} % Default to the plain heading style until the thesis style is called for the body content

%----------------------------------------------------------------------------------------
%	TITLE PAGE
%----------------------------------------------------------------------------------------

\begin{titlepage}
\begin{center}

\vspace*{.06\textheight}
{\scshape\LARGE \univname\par}\vspace{0.5cm} % University name
\textsc{\Large \facname}\\[1.5cm] % Facultry name
\Large{\emph{Draft}}\\[0.5cm] % Thesis type

\HRule \\[0.4cm] % Horizontal line
{\huge \bfseries \ttitle\par}\vspace{0.4cm} % Thesis title
\HRule \\[1.5cm] % Horizontal line
 
\begin{minipage}[t]{0.4\textwidth}
\begin{flushleft} \large
\emph{Author:}\\
\href{http://www.hubbit86.com}{\authorname} % Author name - remove the \href bracket to remove the link
\end{flushleft}
\end{minipage}
\begin{minipage}[t]{0.4\textwidth}
\begin{flushright} \large
\emph{Director:} \\
\href{https://www.cs.upc.edu/~belanche/}{\supname} % Supervisor name - remove the \href bracket to remove the link  
\end{flushright}
\end{minipage}\\[4cm]

\vfill

\large \degreename\\ % Speciality name
\large Specialization:  Computing\\[0.5cm] % Speciality name

\vfill

% \large \textit{A thesis submitted in fulfillment of the requirements\\ for the degree of \degreename}\\[0.3cm] % University requirement text
% \textit{in the}\\[0.4cm]
% \groupname\\\deptname\\[2cm] % Research group name and department name
\includegraphics{img/logo.pdf} % University/department logo - uncomment to place it

\vfill

{\large \today}\\[4cm] % Date

\vfill
\end{center}
\end{titlepage}

\newcommand{\tabhead}[1]{\textbf{#1}}
\newcommand{\VS}{\vspace{6pt}}
\newcommand{\vt}[1]{\vec{#1}}
\newcommand{\st}{\ |\ }
\newcommand{\vb}[1]{\mathbf{#1}}
\newcommand{\T}{\text{T}}
\newcommand{\R}{\mathbb{R}}
\newcommand{\ip}[1]{\langle#1\rangle}
\newcommand{\mrk}[1]{\cellcolor{gray!25}#1}
\newtheorem{lemma}{Lemma}

%----------------------------------------------------------------------------------------
%	ABSTRACT PAGE
%----------------------------------------------------------------------------------------

\begin{abstract}
	%\addchaptertocentry{\abstractname} % Add the abstract to the table of contents

	\centering
	\parbox{0.8\textwidth}{
		\textbf{English:} In machine learning, feature selection algorithms such as SVM-RFE are used to find a subset of statistically relevant features. In this project, we propose multiple extensions of this algorithm, to try to improve its performance or computational cost. The extensions we've tried, include non-lineal kernels, internal sampling and dynamic step.
	}\par
	\vspace{0.4cm}
	\parbox{0.8\textwidth}{
		\textbf{Spanish:} En aprendizaje automatico, algoritmos como SVM-RFE son usados para encontrar subconjuntos de caracteristicas estadisticamente relevantes. En este proyecto proponemos multiples extensiones para este algoritmo, con el objectivo de encontrar mejoras en su acierto o su coste computacional. Las extensiones que hemos provado inlcuyen kernels no lineales, muesteo interno y paso dinamico.
	}\par
	\vspace{0.4cm}
	\parbox{0.8\textwidth}{
		\textbf{Catalan:} En apranentatge automatic, algoritmes com el SVM-RFE son utilitzats per trobar sunconjunts de carecteristiques estatisticament rellevants. En aquest projecte proponem multiples extensions per aquest algoritme amb l'objectiu de trobar millores en el seu encert o el seu cost computacional. Les extensions que hem probat inclouen kernels no lineals, mostreig intern i pas dinamic.
	}

\end{abstract}

%----------------------------------------------------------------------------------------
%	SYMBOLS
%----------------------------------------------------------------------------------------

\begin{symbols}{lll} % Include a list of Symbols (a three column table)

	$n$ & Number of observations & \\
	$m$ & Number of dimensions & \\

	\addlinespace

	$X$ & Dataset & $\{\vt{x_1}, \vt{x_2}, \dots, \vt{x_n}\}$ \\
	$\vt{x_i}$ & Example, observation, ... & $\{x_1, x_2, \dots, x_d\}$ \\
	$Y$ & Dataset labels & $\{y_1, y_2, \dots, y_n\}$ \\
	$\vt{w}$, $\vb{w}$ & Weight vector & $\{w_1, w_2, \dots, w_d\}$ \\
	$b$ & Constant term, bias & E.g. $\vt{w} \cdot \vt{x} + b = 0$\\
	$\vt{u} \cdot \vt{v}$ & Dot/Scalar product (vector) & $u_1v_1 + u_2v_2 + \dots + u_nv_n$ \\
	$\vb{u^\T}\vb{v}$ & Dot/Scalar product (matrix) & $u_1v_1 + u_2v_2 + \dots + u_nv_n$ \\

	\addlinespace

	$\alpha_i, \gamma_i$ & Lagrange multipliers & \\
	$\xi_i$ & Slack variables & \\
	$C$ & Regularization parameter & \\

	\addlinespace 

	$\ip{\vb{u}, \vb{v}}$ & Inner product & E.g. $\vt{u} \cdot \vt{v}$ \\
	$|x|$ & Absolute value & $\sqrt{x^2}$\\
	$||\vt{u}||$ & Euclidean length & $\sqrt{\vt{u} \cdot \vt{u}}$\\
	$d(\vt{u}, \vt{v})$ & Euclidean distance & $||\vt{u} - \vt{v}||$\\

	\addlinespace

	$\phi(\vt{x})$ & Feature map & $\phi : D \rightarrow H$\\
	$k(\vb{x_i}, \vb{x_j})$ & Kernel function & $\ip{\phi(\vb{x_i}), \phi(\vb{x_j})}$\\
	$\gamma, \sigma$ & RBF Kernel Parameter & \\
	$\vb{H}$ & Hessian Matrix of dual SVM & $\vb{H_{i,j}} = y_iy_jk(\vb{x_i}, \vb{x_j})$\\

	\addlinespace

	$\vt{s}$ & Surviving features & $[1,2, \dotsc, d]$ \\
	$t$ & Constant Step & \\
\end{symbols}

%----------------------------------------------------------------------------------------
%	THESIS CONTENT - CHAPTERS
%----------------------------------------------------------------------------------------

\mainmatter % Begin numeric (1,2,3...) page numbering

\pagestyle{thesis} % Return the page headers back to the "thesis" style
\tableofcontents
% Chapter 1

\chapter{Introduction} % Main chapter title

\label{Chapter1} % For referencing the chapter elsewhere, use \ref{Chapter1} 

%----------------------------------------------------------------------------------------

This bachelor thesis at the Computer Engineering Degree, specialization in Com\-puting, has been done in the Facultat d’Informàtica de Barcelona of the Universitat Politècnica de Catalunya (UPC) and directed by Luis Antonio Belanche Muñoz, PhD. in Computer Science.

\section{Context}
\label{sec:context}

In statistics, machine learning, data-mining, and other related disciplines, it is often the case that there are redundant or irrelevant data in a dataset\footnote{A table with rows / records / observations and columns / variables / features / dimensions / predictors / attributes.}. Indeed, before we can start working with the data, some form of data analysis and cleaning is required. Data cleaning may include removing duplicated rows or rows with missing values, removing observations that are clearly outliers, removing irrelevant variables (e.g. name, surname, email address), etc.

With the new era of Big Data, datasets have increased in size, both in number of observations and in dimensions. Applying classical data-mining and machine learning algorithms to these high-dimensional data raise multiple issues collectively known as “the curse of dimensionality”. One such issue is the elevated, usually intractable, cost and memory requirements derived from the non-linear (on number of dimensions and observations) complexities of the algorithms. Another issue has to do with data in a high-dimensional space becoming sparse and negatively af\-fect\-ing the performance of algorithms designed to work in a low-dimensional space. And finally, a third issue is that with a high number of dimensions the algorithms tend to overfit, that is, they don't generalize enough and end up producing models that perform worse with real data than their predicted performance with the training data. (\cite{li_feature_2017})

Simple manual data cleaning is not enough to achieve satisfactory amounts of dimensionality reduction. In this case we can use automatic techniques. We can classify such techniques in two categories: feature extraction and feature selection.

\subsection{Feature Extraction}

Feature extraction techniques transform the original high-dimensional space into a new low-dimensional space by extracting or deriving information from the original features. The premise is to compress the data in order to pack the same information at the expense of model explainability\footnote{The ability to explain why certain predictions are made. Also, interpretability.}. Continuing with our data compression analogy, virtually all feature extraction techniques perform lossy com\-pres\-sion. That is, some information is lost which makes the process irreversible. 

Some well known feature extraction algorithms include Principal Component Analysis (PCA) (\cite{scholkopf_kernel_1997}) and auto-encoders (\cite{vincent_extracting_2008}), the first being a transformation over the feature-space and the second a neuronal network. PCA may be extended with a kernel method in order to make non-linear transformations. Similarly, we will also use non-linear kernels in some of our extensions.

\subsection{Feature Selection}
\label{sec:ch1.feature_selection}

In contrast, feature selection only selects a subset of the existing features, ideally the most relevant or useful. This may imply a greater loss of information compared to feature extraction, but it doesn't reduce explainability. Some problems require feature selection explicitly. In domains such as genetic analysis and text mining, feature selection is not necessarily used to build predictors. For example in micro\-array analysis feature selection is used to identify genes (i.e. features) that dis\-criminate between healthy and sick patients. 

Note that the number of possible selections of all sizes is $2^m$. This is the same as the amount of subsets of a set with size $m$, where $m$ is the amount of total features. The number of possible selections for some feature subset size $k$, is then ${m \choose k} = \frac{m!}{k!(m - k)!}$. This makes the optimal feature selection via exhaustive search intractable. Instead, we use methods that do not require an exhaustive search, such as a greedy algorithm, at the expense of not being able to guarantee an optimal solution.

Feature selection methods may be classified by how they are constructed in three categories:

\begin{itemize}
    \item \textbf{Filters:} A \emph{feature ranking criterion} is used to sort the features in order of rel\-e\-vance, then select the $k$-most relevant.
    \item \textbf{Wrappers:} They use a learning machine (treated as a black box) to train and validate the dataset with different subsets of variables. They rank the subsets based on the performance (score) of the model. A wide range of learning machines and search strategies can be used. Within the greedy strategies we find \emph{forward selection} and \emph{backward elimination}.
    \item \textbf{Embedded:} Like wrapper methods but more efficient. They use information from the model itself at training time to make feature selection. Because they don't use the score, they can also skip testing the model.
\end{itemize}

In both wrapper and embedded methods greedy strategies can be used. \textbf{SVM-RFE} is a feature selection algorithm, of the embedded class, that uses Support Vector Machines (SVM) and a greedy strategy called Recursive Feature Elimination (RFE). This algorithm is an instance of backward elimination. It starts with a set of all features and eliminates the less relevant each iteration. Within each iteration SVM-RFE behaves like a filter method and uses a feature ranking criterion to decide which feature to eliminate. SVM-RFE takes advantage of the fact that, for linear SVM, the ranking criterion is to take the variable with the smallest absolute weight in each iteration. Where the weights are the co\-ef\-fi\-cients of the hyperplane resulting from the SVM training.  (\cite{guyon_introduction_2003})

%----------------------------------------------------------------------------------------

\section{State of the art}

The SVM-RFE algorithm was first proposed in a paper on the topic of cancer class\-ification (\cite{guyon_gene_2002}). This paper uses the SVM-RFE algorithm to identify highly discriminant genes, encoded as features, that have plausible relevance to cancer diagnosis. Since then, SVM-RFE has remained a popular technique for gene selection and the original paper cited more than four thousand times.

The paper already proposes some natural extensions to the al\-go\-rithm, such as eliminating multiple features each iteration based on the ranking and a \emph{step} constant. Further research has been on improving and extending different parts of the al\-go\-rithm. These include the use of a Gaussian kernel (\cite{xue_nonlinear_2018}), using multiple SVM in the same iteration (\cite{wang_classification_2011}) or simply trying to find a better ranking criterion (\cite{mundra_svm-rfe_2007}).

%----------------------------------------------------------------------------------------

\section{Project goals}
\label{sec:ch1.objective}

The main objective of this project is to research possible extensions and im\-prove\-ments that try to optimize the SVM-RFE algorithm. Optimizations may be in the form of improved performance or a reduction in time utilization. We do not propose here any optimizations that may affect space complexity. As such, we also skip the space complexity analysis for the whole project. We've classified the possible extensions as follows.

\subsubsection*{Main extensions}

These focus on improving the performance of the selection of features.

\begin{itemize}
    \item \textbf{Non-linear kernel:} Use a more general ranking criterion and apply it to handle SVM with arbitrary kernels.
    \item \textbf{Multi-class criteria:} Find a ranking criterion that can handle multiple weight vectors in a useful way.
\end{itemize}

\subsubsection*{Secondary extensions}

Focus on reducing the computational cost.

\begin{itemize}
    \item \textbf{Internal sampling:} Use a different subset of the observations on each iteration.
    \item \textbf{Dynamic step:} Instead of using a constant value in each iteration, calculate it dynamically.
    \item \textbf{Stop condition:} Determine the amount of features that are relevant (SVM-RFE only provides a ranking) in an effective and inexpensive manner.
\end{itemize}

\subsubsection*{Combination of various extensions}

\begin{itemize}
    \item \textbf{Combo:} Mix \emph{internal sampling}, \emph{dynamic step} and \emph{non-linear Kernels} into a single implementation.
\end{itemize}

\subsection{Goals break down}

To accomplish these goals, the project has been sub\-divided in two parts, each having specific tasks for each extension:

\subsubsection*{Algorithmic Part}

\begin{itemize}
    \item Do research in SVM-RFE and in the extensions that will be tackled in the project.
    \item {
        For each extension:
        \begin{itemize}
            \item Design the algorithm and write its formalization in pseudocode.
            \item Define the expected advantages or disadvantages of this extension over the base SVM-RFE.
            \item Compute the time complexity.
        \end{itemize}
    }
\end{itemize}

\subsubsection*{Practical Part}

\begin{itemize}
    \item Program the base SVM-RFE algorithm and the extensions.
    \item {
        For each extension:
        \begin{itemize}
            \item Analyze its behavior for artificial data sets.
            \item Analyze its behavior for real-world data sets.
        \end{itemize}
    }
    \item Compare the results obtained with the ones expected.
    \item Combine multiple extensions to further improve the algorithm.
    \item Draw conclusions about all the results obtained in the project.
\end{itemize}

%----------------------------------------------------------------------------------------

\subsection{Stakeholders}

This project is intended to be of use for many involved parties. The most directly involved group, is the tutor and the researcher. Luis Antonio Belanche Muñoz is the tutor of this project. Robert Planas Jimenez would be the researcher. Feature selection algorithms is one of the areas of research of the tutor, and he has wanted to explore extensions to the SVM-RFE algorithm. He will lead and guide the researcher for the correct development of the project. The researcher is responsible for planning, developing and documenting the project, as well as experimenting, analyzing and drawing conclusions.

The other group of interested parties would be stakeholders that do not interact with the project directly but still benefit from it. In the first place we have researchers on the fields of bioinformatics and data mining, that use machine learning methods (specifically, SVM-RFE) for micro-array analysis, text analysis, or other of its popular applications. Indirectly, companies that make use of any findings will also benefit. Finally, the general population may also benefit from better diagnostics and more effective drugs. 

\label{sec:risk}
\subsection{Potential obstacles and risks}

Some obstacles and risks identified that could potentially prevent the correct exec\-ution of the project are:

\begin{itemize}
    \item \textbf{Deadline of the project:} There is a deadline for the delivery of the project. This being a research project however, is considerably hard to estimate how much time tasks will take, or even decide whether a task has been finished or not.
    \item \textbf{Bugs on some libraries:} This is considered of low risk, but is still a possibility that errors on the software package used extend to code, making it work in\-correctly.
    \item \textbf{Insufficient computational power:} Machine learning algorithms, in general, can be very resource intensive. It could be the case that our hardware can not handle some datasets. 
    \item \textbf{Hardware related issues:} A hard drive failure could occur that would end in lost data, or a failure in a router could disconnect us from the internet.
    \item \textbf{Health related issues:} In addition to health issues that can occur at any time without prior notice, we're in the middle of a pandemic.
\end{itemize}


\section{Methodology}

\subsection{Framework}

The methodology that we will use for the project is a combination of Kanban (\cite{matharu_empirical_2015}) and Waterfall (\cite{mahadevan_running_2015}) methodologies. Waterfall will be used to define the general phases of the project, and Kanban for tracking the individual tasks. In Waterfall, tasks can not start until the previous task has been completed, and thus following strict deadlines is important. Phases will not start until the previous phase ends. Each phase will be composed of multiple tasks, which will then be managed by the Kanban method\-ol\-o\-gy.

Kanban is much more flexible than Waterfall. Its principal objective is to manage tasks in a general way, by assigning different statuses to them. Kanban stands out by its simplicity, and will continue to endorse that simplicity by managing the visual representation of the cards in a simple, plain, text formatting. Each card will be in a row, with the first column defining its name and the other its status. The statuses we've considered are:

\begin{itemize}
    \item \textbf{To do:} A basic idea of the task is present.
    \item \textbf{Definition:} The task is in the theoretical part.
    \item \textbf{Implementation:} The task is in the practical part.
    \item \textbf{Completed:} The task is finished.
\end{itemize}

An uppercase letter \texttt{"X"} will mark the current state of any given card. If more granular information is required, other marks may be used instead. For example, to indicate that the task is paused a \texttt{"P"} would be used. To indicate the progress within some stage a percentage would be used. For easy monitoring, this table will be kept in the \texttt{readme.md} file of our \texttt{GitHub} repository.

\subsection{Validation}

We will use a GitHub repository as a tool for version control, which will allow us to share code easily and recover from data lose. The repository will contain both the code for the experiments, each in one subfolder, and code for the documentation. In order to verify the implemented code, it will be tested with multiple data sets. In the practical part, hyper-parameters will be selected using some model validation technique, such as the cross-validation. Each ex\-per\-i\-ment will be done at least 6 times and the average of the results will be the final result. 

Face-to-face meetings with the tutor of the project will be scheduled once every two weeks. In these meetings the current project status will be discussed, and the tasks to do during the following two weeks will be defined. In case of unexpected problems, extraordinary meetings can be arranged.

\include{chap/ch2_plan}
\include{chap/ch3_budget}
\chapter{Background}
\label{Chapter4}

In this section we're going to review the concepts required to understand the SVM-RFE algorithm. Section \ref{sec:context} already introduced some concepts around the SVM-RFE algorithm, placed it in context, and enumerated some of its applications. In this section we'll focus on the inner workings of the algorithm and how these parts add together. 

\section{Machine learning}

Machine learning is a subfield in the broader discipline that is artificial intelligence, with a particular take in statistics. These algorithms, also called learning machines or just machines, use data to learn patterns and make predictions. The data is collected in a separated unrelated process, and structured in the form of a \emph{dataset} (a collection of data). Once the dataset is further cleaned and prepared, the learning machine finally consumes it in a process called \emph{training}. After this process the machine pro\-duces a \emph{model}, which is a function that can be used to make predictions on new data.

Two canonical problems in machine learning are regression and classification problems.

\subsection{The dataset}

A dataset is simply a collection of data. In the context of machine learning this data-set will be used to make predictions, take decisions, or find patterns. For a machine learning algorithm to be able to consume a dataset the first step is to represent it in tabular form. 

Most datasets come already in tabular form. Some of the most notable ex\-cep\-tions are datasets involving images. In this case computer vision methods are often used to extract numeric data representing characteristics of the image. Sometimes a more direct transformation can also be made, for example making each feature be the intensity level of a single pixel in the image. This of course produces a high number of features, most of which are redundant or irrelevant (e.g. features representing pixels in the background). 

In a dataset represented as a table, columns describe different \emph{features}, \emph{properties} or \emph{attributes} of some group of objects and rows represent \emph{instances} or \emph{examples} of that group. For example, if objects were vehicles then features could include the brand, power, weight, max\-imum speed, and other such characteristics of various vehicles, each of which would be in a row. Different names are used in different contexts. One of the most typical naming conventions comes from the statistics domain which refers to columns as \emph{variables} and rows as \emph{observations}. Sometimes different nomenclature is used to differentiate features in different stages of the cleaning and preparation process, but in this thesis we use them all instinctively.

\begin{table}[h]
    \makebox[\textwidth][c]{
        \begin{tabular}{l l l l l l l l l}
        \toprule
        \tabhead{Src} & \tabhead{Dst} & \tabhead{NAT-Src} & \tabhead{NAT-Dst} & \tabhead{Action} & \tabhead{Sent (B)} & \tabhead{Rcvd. (B)} & \tabhead{Packets} & \tabhead{Elapsed (sec)}\\
        \midrule
        57222 & 53 & 54587 & 53 & allow & 94 & 83 & 2 & 30 \\
        56258 & 3389 & 56258 & 3389 & allow & 1600 & 3168 & 19 & 17 \\
        6881 & 50321 & 43265 & 50321 & allow & 118 & 120 & 2 & 1199 \\
        43537 & 2323 & 0 & 0 & deny & 60 & 0 & 1 & 0 \\
        50002 & 443 & 45848 & 443 & allow & 6778 & 18580 & 31 & 16 \\
        \bottomrule\\
        \end{tabular}
    }
    \caption{Example dataset extracted from the Internet Firewall Data (\cite{ertam_internet_2019}). Only five observations and main variables shown. }
    \label{tab:example_dataset}
\end{table}

\subsection{Classification}
\label{sec:ch4.classification}

For the classification problem we want to predict in witch group or \emph{class} to assign some new observation. Table \ref{tab:example_dataset} provides an example of what could be a good classification problem. Imagine we are building a firewall and want it to predict if some new packet in the network should be allowed to pass. We could train a classification learning machine with the dataset represented in the table (the full version) and produce a model capable of doing such predictions. Although it is trivial to identify a pattern in the example, it may not be so for real world examples. Automatic \emph{pattern recognition} is key in solving many real world problems, and it is one of the features of these algorithms.

Mathematically, the set of observations is defined as $X = \{\vt{x_1}, \vt{x_2}, \dots, \vt{x_n}\}$ and the set of the corresponding labels or target classes is $y = \{y_1, y_2, \dots, y_n\}$ with every element of this set being some class $y_i = C_k$ of a discrete set of classes with size $K$. They are called \emph{domain set} and \emph{label set} respectively. For our example, the classes would be $C = \{\text{allow}, \text{deny}\}$. Thus, we can describe the goal of a classification problem as assigning some class $C_k$ to an input vector $\vt{x}$.

Although we've used strings to represent classes, part of the preparation process of the dataset involves turning all values into numerical scalars, so our classes would actually be some natural numbers. Also notice that every $\vt{x_i}$ is a vector containing a single numerical value for each feature excluding the label. 

We make a distinction between two-class problems (or binary) and multi-class problems. Two-class problems typically use $C = \{0, 1\}$ as classes and thus can be modeled with a boolean function or, if a probability is desired, a function that returns values between 0 and 1. This simplifies the model substantially, in fact some learning machines such as the SVM can only work with two-class problems, and use different methods to extend to the multi-class version.

\subsection{Visualization}
\label{sec:ch4.visualization}

A visualization of the problem in some euclidean space is required to understand how most classification algorithms work, and in particular SVM. Typically, clas\-si\-fi\-cation models divide the input vector space\footnote{The euclidean space of minimum dimension containing all possible input vectors $\vt{x}$.} into \emph{decision regions}. The boundaries of such regions are called \emph{decision boundaries} or \emph{decision surfaces}. If the decision boundary is in the form of some hyperplane of dimension $(D - 1)$, with $D$ being the dimension of the space, then we say that it's a linear model. A dataset whose classes can be completely separated by such linear decision boundary is said to be \emph{linearly separable}.

\begin{figure}[H]
    \centering
    \begin{subfigure}[b]{0.4\linewidth}
        \includegraphics[width=\linewidth]{img/ch4/nolinsep.png}
        \subcaption*{Not linearly separable}
    \end{subfigure}
    \begin{subfigure}[b]{0.4\linewidth}
        \includegraphics[width=\linewidth]{img/ch4/linsep.png}
        \subcaption*{Linearly separable}
    \end{subfigure}
    \caption{Decision regions and 1-D hyperplane boundary of some linear model for two datasets. Points are observations, colors indicate the class.}
    \label{fig:ch4.sep}
\end{figure}

Given a boundary we can determine in which region some new point $\vt{x}$ falls using a \emph{discriminant function}. One of the most trivial cases is when the boundary is lineal. It is convenient now to do a small refresh on the equation that describes a hyperplane.

\subsubsection*{Equation of a line (slope-intercept form)}
\begin{align}
    y = mx + t
\end{align}

Where $m$ is the \emph{slope} or \emph{gradient}, $x$ is the independent variable of the function $y = f(x)$ and $t$ is the y-intercept value, the point of the function where the line crosses the y-axis, i.e. $t = f(0)$. This description has the advantage that can be directly represented as a function $f(x) = mx + t$.

\subsubsection*{Equation of a line (standard form)}
\begin{align}
    ax + by = c
\end{align}

This is an equivalent form, this one however is not representable by a function $f: \mathbb{R} \rightarrow \mathbb{R}$, instead it is often represented as a set $L = \{(x, y) \st ax + by = c\}$. This allows representing vertical lines and also features the useful property that $(a, b)$ is the normal vector\footnote{A vector that is perpendicular to the surface.} of the line.

\subsubsection*{Equation of a line (general form)}
\begin{align}
    ax + by - c = 0
\end{align}

A simple linear transformation of the standard form produces the general form (\cite{noauthor_wikipedia_2021}). This conserves the normal vector and also has the advantage of being representable by an implicit function\footnote{A function of multiple variables $f: \mathbb{R}^n \rightarrow \mathbb{R}$ such that we only consider solutions where $f(X) = 0$. For example a circle can be defined with an implicit function $f(x, y) = x^2 + y^2 - r$, but if we were to plot it in 3-D we would instead see an inverted cone.}. Notice that if represented in 3-D it would produce a plane that is perpendicular to the $xy$ plane, since its normal would always have the form $(a, b, 0)$.

\subsubsection*{Equation of a 3-D Plane}
\begin{align}
    ax + by + cz + d = 0
\end{align}

We can extend the general form of a line with one more dimension, it only requires adding the new term $cz$ for the new dimension. We've also changed the sign of the constant term $d$ for compactness. This is a conceptual move, not an algebraic one.

\subsubsection*{Equation of a Hyperplane}
\begin{align}
    w_1x_1 + \dots + w_nx_n + w_0 = 0
\end{align}

This is a generalization of the equation of a 3-D plane for $n$ dimensions. It also happens to be the definition of a \emph{linear equation}. The variables $w_1, \dots, w_n$ are called \emph{coefficients}, \emph{parameters} or \emph{weights}, and the variable $w_0$ is the constant term. It is important to avoid confusing $x_i$ with $\vt{x_i}$, the first is the one the coordinates of a point in some dimension, and the other is an observation of a dataset (a point). In particular, it may be the case that $\vt{x_j} = (x_1, x_2, \dots, x_n)$.

We may want to compact this expression more by using vectors. So, another form for representing the equation of a hyperplane is:
\begin{align}
    \vb{w}^{\T}\vb{x} + w_0 = 0
\end{align}

Where $\vb{w}$ is called the \emph{vector weight} and $w_0$ the \emph{bias}. We can turn this equation into a discriminant function for two classes by simply considering what happens with points that are not in the hyperplane. By definition such points meet one of the two inequations:
\begin{align*}
    \vb{w}^{\T}\vb{x} + w_0 > 0 \\
    \vb{w}^{\T}\vb{x} + w_0 < 0
\end{align*}

A point will be a solution of one of these inequations depending on whether it is in the subspace, i.e. discriminant region, facing the direction of the normal or the opposite.

A linear binary classification learning machine is thus an algorithm that given some dataset finds appropriate values for the parameters $\vb{w}$ and $w_0$ of the hyper\-plane in order to produce a discriminant function $y : \mathbb{R}^n \rightarrow \mathbb{R}$ such as:
\begin{align}
    y(\vb{x}) = \vb{w}^{\T}\vb{x} + w_0
\end{align}

\subsection{Performance}
\label{sec:ch4.performance}

Usually models produced by a learning machine do not always make correct pre\-dict\-ions, instead we consider them good enough if they can classify new data correctly most of the time. In order to quantize how good of a predictor some model is we can use various performance metrics.

The most typical metric for classification problems is \emph{accuracy}. This is the ratio of correct predictions versus the total number of predictions made. The inverse to the accuracy is defined as the \emph{error}. Notice that the classification accuracy will always be some percentage above 50\%. This is because a classifier performing consistently worse than that can be turned into a good classifier by simply flipping the output of the discriminant function. Thus, the worst possible classifier is that of a coin toss, i.e. a uniform random distribution, with an expected accuracy of exactly 50\%.

In some problems a distinction is made between misclassifications depending on which class is misclassified. An example of this is how misclassifying a patient with cancer with a healthy diagnosis is worse than misclassifying a healthy patient with a cancer diagnosis, the consequence of one is potentially death due to lack of treatment while the other is simply more investigation. For these problems a quadratic amount of cases appears. For the binary classification the standard nomenclature is to name the classes \emph{positive} and \emph{negative} and then prefix them with \emph{true} or \emph{false} depending on whether the prediction was correct or not. In this way a matrix called \emph{confusion matrix} is created, with the diagonal containing all the correct predictions.

\begin{figure}[H]
    \centering
    \includegraphics[width=0.5\linewidth]{img/ch4/confusion.png}
    \caption{Confusion matrix extracted from the paper “Automatic Device Classification from Network Traffic Streams of Internet of Things” (\cite{bai_automatic_2018}).}
    \label{fig:ch4.confusion}
\end{figure}

From a confusion matrix various other metrics can be extracted, such as \emph{precision}, \emph{specificity} or \emph{recall}, but it is unlikely that we make use of them in this project. Often models internally use a \emph{loss} or \emph{cost} function that they try to minimize in order to optimize the parameters, the inverse of which is called \emph{utility function}. Even if the model doesn't internally use one such function, one can be constructed form performance metrics, which is then referred as the \emph{score}. 

It is known that learning requires both \emph{generalization} and \emph{memorization}. If a model memorizes the dataset, thus has high accuracy on data already in the dataset, but doesn't generalize well, thus has low accuracy when predictions are done on new data, we say there is \emph{overfitting}. Notice that for this to be detected we must test the dataset with new data. In order to do so a dataset is often split in \emph{train} and \emph{test} subsets, and although accuracies form both subsets may be reported, only the accuracy of the test subset may be taken as good.

One of the possible causes of overfitting is the \emph{Bias-Variance} trade-off. It's an effect in which if you select a very simple model then the algorithm fails to generalize (bias) but selecting a very complex algorithm increases memorization and leads to high variability in the presence of an unseen observation (variance). The best model is thus one with a middle-ground complexity.

Selecting the complexity of an algorithm can be done with the use of some extra parameters, not fitted by the training phase of the leaning machine, that we call \emph{hyper-parameters}. The existence of this extra parameters depends on the models, some may not have any. Because these parameters modify the model, searching for the best hyper-parameters is called \emph{model selection}. Model selection is often done in a brute force manner, by simply trying different values of the hyper-parameters and selecting the ones that give the best result. Al\-though this coarse strategy is usually enough, more complex search strategies such as successive halving or genetic algo\-rithms (\cite{claesen_hyperparameter_2015}) can also be used. In the case of one single hyper-parameter we can draw a plot and visualize how the score evolves. Because model selection may be understood as some kind high level training, specially when com\-plex search strategies are used, a third division on the dataset may be made, named the \emph{validation} set.

\begin{figure}[H]
    \centering
    \includegraphics[width=0.5\linewidth]{img/ch4/bias-and-variance.png}
    \caption{Bias-Variance trade-off. Source (\cite{bisong_machine_2021}).}
    \label{fig:ch4.biasvariance}
\end{figure}

Machine learning algorithms work better the more observations they are trained with, however it is not often the case that we have an unlimited amount of them. In analyzing the performance of a leaning machine, or doing model selection, it may be useful to repeat the experiments multiple times with different data of the same distribution, i.e. from the same dataset. This provides second order statistics, such as the mean or the covariance, on the resulting scores, which may prove very useful to get a better idea of what is going on. This however implies further slicing the dataset in as many slices as experiments one wants to do. After so much slicing, the remaining training subsets would often contain very few observations, thus reducing the performance of the models and nullifying the advantages of having a probability distribution on the scores. 

A method called \emph{cross-validation} can be used to make mul\-ti\-ple experiments with\-out increasing the amount of data. The method first makes $K$ partitions called \emph{folds}. For each fold, one experiment is made such that the fold is used as the validation set and the remaining folds become the training set. This allows a portion $(K-1)/K$ of the observations to be used for training in each run and allows assessing performance using the whole dataset. In the particular case where $K = N$, useful for when the amount of observations is really low, we name this method \emph{leave-one-out} cross-validation. It is also possible to shuffle the data, which allows even more re-utilization. Although this method is computationally expensive (since it requires running the whole experiment $K$ times), it has the advantage of being trivial to parallelize, and thus it can run at increased speeds in computers with a multi-core CPU architecture. 

\section{Support Vector Machines}

As it has been discussed in section \ref{sec:ch4.visualization}, the decision boundary of a binary class\-ification prob\-lem may be represented with a hyperplane, and it is the weights of this hyperplane what the learning machine tries to fit such that the hyper\-plane correctly separates the two classes. There may be infinitely many hyperplanes that correctly separate the examples of two classes, but some of them are better suited than others. For the purpose of generalization, we want decision regions that can accommodate new data outside the convex hulls\footnote{In geometry, the \emph{convex hull} or \emph{convex envelope} of a shape is the smallest convex set that contains it.} of each group of examples. That is, we want some kind of separation between the hyperplane and the observations.

\begin{figure}[H]
    \centering
    \begin{subfigure}[b]{0.4\linewidth}
        \includegraphics[width=\linewidth]{img/ch4/boundrybad.png}
        \subcaption*{Bad boundary}
    \end{subfigure}
    \begin{subfigure}[b]{0.4\linewidth}
        \includegraphics[width=\linewidth]{img/ch4/boundrygood.png}
        \subcaption*{Good boundary}
    \end{subfigure}
    \caption{Showing how one decision boundary may be better than another. \texttt{X} filled in blue represents new data.}
    \label{fig:ch4.sep}
\end{figure}

There exist various learning machines that accomplish this separation, e.g. the \emph{Fisher’s linear discriminant}. Another such machine is the Support Vector Machine (SVM), which is based on the idea of finding the biggest margin between the extreme points of each class and setting the decision boundary in the center of such margin. For this reason SVM machines are said to be \emph{Maximum Margin Classifiers}.

\begin{figure}[H]
    \centering
    \begin{subfigure}[b]{0.4\linewidth}
        \includegraphics[width=\linewidth]{img/ch4/mmc_good.png}
        \subcaption*{Maximum margin}
    \end{subfigure}
    \begin{subfigure}[b]{0.4\linewidth}
        \includegraphics[width=\linewidth]{img/ch4/mmc_bad.png}
        \subcaption*{Quite good margin}
    \end{subfigure}
    \caption{Two quite good decision boundaries with a margin, a maximum margin classifier would choose the left one.}
    \label{fig:ch4.sep}
\end{figure}

The \emph{margin} is defined as the perpendicular distance (i.e. in the direction of the normal) between the decision bound\-ary and the closest of the examples. There is only one decision boundary that maximizes the margin. Only a limited amount of examples, defined by the number of dimensions, will reach the limits of that margin (and thus be the closest), these are called the \emph{support vectors}.

\subsection{Discriminant function}

We've already seen how to we can build a discriminant function from the equation of the hyperplane. Here we will explore this in a bit more depth and take a different approx. 

Given a vector of unknown length $\vt{w}$ that is perpendicular to the hyperplane (thus the normal) such that its origin is at the origin of the coordinate system. And given another vector $\vt{x}$ also with origin at the origin of the coordinate system. Notice that, by how we are defining them, they are actually points, but we may want to think of them as vectors for now. We are interested in knowing in which decision region the point $\vt{x}$ is placed.

To do so, we make the dot product between $\vt{w}$ and $\vt{x}$, which, in particular, will give us a scalar corresponding to the length of the projection of $\vt{x}$ over $\vt{w}$. Notice that for points within the hyperplane that length will be some constant value $c$. Thus, we can know if point $\vt{x}$ is in one region or another by comparing it with $c$. This can be expressed with the inequality $\vt{w} \cdot \vt{x} \ge c$. By making a variable $b = -c$ and expressing the dot product as a product of matrices, we can conveniently rearrange this equation as the decision rule:
\begin{align}\label{eq:drule}
    \vb{w}^\T \vb{x} + b \ge 0
\end{align}
This is analogous to the decision rule found in section \ref{sec:ch4.visualization}, in fact if we make an equality instead of an inequality we discover the equation of the hyperplane, from there we could do the process in reverse until we find the general equation of a line. 

\subsection{Margin formalization}

Notice from equation \ref{eq:drule} that the constant value of $b$ depends directly on the length $||\vt{w}||$, and because this length is not limited (there are infinite vectors that are normal to the hyperplane), an infinite amount of combinations exists.

Therefore, we may want to restrict the decision rule to a single combination. Also, it is convenient that this restriction takes such a form that allows defining the margin. First we assign to our classes the numerical values $C = \{-1, 1\}$. Then we impose the condition that points in a decision region outside the margin must have values greater than $\pm 1$, thus for points $\vb{x}_{(+)}$ in class “$1$” and $\vb{x}_{(-)}$ in class “$-1$” we get:
\begin{align*}
    \vb{w}^\T \vb{x}_{(+)} + b &\ge 1\\
    \vb{w}^\T \vb{x}_{(-)} + b &\le -1
\end{align*}

This pair of equations can be simplified to a single equation by considering the vector $\vt{y}$ that we defined in section \ref{sec:ch4.classification}. Because of the numerical values we've assigned to the classes, it will contain elements such that every $y_i \in \{-1, 1\}$. Notice that if we multiply both sides by $y_i$ (and given that we know what the value of $y_i$ will be in each case), we can produce the single equation:
\begin{align}\label{eq:svmconstrain}
    y_i (\vb{w}^\T \vb{x_i} + b) \ge 1
\end{align}

From this equation we may find the values of the hyperplanes that border the margin (named \emph{gutter}), that is, the points where the last equation is exactly 1. 

\subsubsection*{Equation of the gutters}
\begin{align}\label{eq:gutters}
    y_i (\vb{w}^\T \vb{x_i} + b) -1 = 0
\end{align}

Although now we know the equation for the gutters (analogy for the margin being a street), what we're really interested about is in the distance between them. We can calculate that distance as the diff\-erence vector form any two points in the gutters projected using the unit vector of $\vt{w}$:
\begin{align*}
    (\vt{x}_{(+)} - \vt{x}_{(-)}) \cdot \frac{\vt{w}}{||\vt{w}||}
\end{align*}

Notice that form equation \ref{eq:gutters} we can derive $\vt{w} \cdot \vt{x}_{(+)} = 1 - b$ and $\vt{w} \cdot \vt{x}_{(-)} = b - 1$, applying the distributive property we obtain:
\begin{align*}
    \frac{(1 - b) - (b - 1)}{||\vt{w}||} \implies \frac{2}{||\vt{w}||}
\end{align*}

Since we've defined the margin to be the distance from the decision boundary to a gutter (i.e. where support vectors are), and since the decision boundary is at the same distance to both gutters, then we can see that the length of the margin will be half the distance between the gutters.

\subsubsection*{Length of the margin}

\begin{align}
    \frac{1}{||\vt{w}||} = \frac{1}{\sqrt{\vb{w}^\T \vb{w}}}
\end{align}

\subsection{Optimization problem}

The problem of finding the greatest margin can be formulated as an optimization problem. Specifically we want to maximize $1/||\vt{w}||$ subject to the constraints defined in equation \ref{eq:svmconstrain}. This is equivalent to minimizing $||\vt{w}||$ subject to the same constraints. For mathematical convenience (which we will see later), we can also divide by 2 and square the objective function without it affecting the optimal solution.

\subsubsection*{Primal form}
\begin{align}\label{eq:svm_primal} 
    \underset{\vt{w}, b}{\text{arg} \min}\ \frac{1}{2} ||\vt{w}||^2 \text{\quad s.t. \quad} \forall i : y_i (\vb{w}^\T \vb{x_i} + b) - 1 \ge 0
\end{align}

This is a constrained quadratic optimization problem. There are various known algorithms that can solve constrained optimization problems, such as the \emph{simplex} algorithm. However, in our case it may be appropriate to use the \emph{Lagrangian} method. This method doesn't directly return a solution, instead it transforms the problem into another version, from which a direct solution may be more easily computed.

\subsubsection*{Lagrangian}

Imagine we had an easier optimization problem. E.g. a maximization pro\-blem such that its optimization function $f : \R^{D-1} \rightarrow \R$ is a function $f(\vt{x})$ of $D$ dimensions, and it has a single equality constrain $g(\vt{x}) = 0$ of $D - 1$ dimensions. Notice how the constraint is “projected” on top of $f(\vt{x})$. Then the Lagrangian method consists on optimizing a new function (equation \ref{eq:lagrangian_base}), via introducing a new pa\-ram\-e\-ter $\lambda$ called the \emph{Lagrangian multiplier}. 
\begin{align}\label{eq:lagrangian_base} 
    L(\vt{x}, \lambda) = f(x) - \lambda g(x)
\end{align}

Notice how there is only a subset of the images of $f(\vt{x})$ for which $f$ and $g$ inter\-sect. Where the intersection is defined as $\{\vt{x} \in \R^{D-1}\ |\ \exists \vt{c} : f(\vt{x}) = c \land g(\vt{x}) = 0\}$. Many values of $c$ contribute to the intersection, however we are only interested in the maximum or minimum values, since for these $\vt{x}$ becomes a \emph{stationary point}.

An interesting property of these functions is that in the stationary points the di\-rec\-tion of the normal of both functions must be the same. Note that if the normal of $f$ was not orthogonal to the surface of $g$, we could increase $c$ by moving a short distance along the constraint surface. Also remember that the normal is the same as the gradient of that function, i.e. $\nabla f$. Therefore, there must exist a parameter $\lambda \neq 0$ (or $\lambda > 0$ for a minimization problem) such that:
\begin{align}
    \nabla f - \lambda \nabla g = 0
\end{align}

From this we see that the constrained stationary condition is obtained by setting $\nabla_x L = 0$. Notice that the gradient is defined as the partial derivatives of each coordinate in the vector, that is:
\begin{align*}
    \nabla_x L = \left( \frac{\partial L}{\partial x_1}, \frac{\partial L}{\partial x_2}, \cdots, \frac{\partial L}{\partial x_n} \right)
\end{align*}

At first glance it may look overwhelming, but given that $L$ is the same for all $x_i$ and that we can reuse the arithmetic operators for operations between vectors, doing the gradient is very similar than doing the partial derivate over a single scalar. 

Going back to the SVM formalization, we can define the Lagrangian of the primal form (equation \ref{eq:svm_primal}) as:
\begin{align}\label{eq:svm_lagrangian_from_primal}
    L(\vt{w}, b, \vt{\alpha}) = \frac{1}{2} ||\vt{w}||^2 - \sum \alpha_i \left[ y_i (\vb{w}^\T \vb{x_i} + b) - 1\right]
\end{align}

Now we want to take the partial derivative with respect to $\vt{x}$ and equal to 0. Note that $||\vt{w}|| = \sqrt{\vb{w}^\T\vb{w}}$, if we elevate this expression to the power of two we can eliminate the square root. The derivative for one component is $\frac{\partial [(1/2)\vb{w}^\T\vb{w}]}{\partial w_i} = w_i$, combining them (derivative of the vector) we get $\vt{w}$. Therefore:
\begin{align}\label{eq:svm_representer}
    \frac{\partial L}{\partial \vt{w}} = \vt{w} - \sum \alpha_i y_i \vt{x_i} = 0 \implies \vt{w} = \sum \alpha_i y_i \vt{x_i}
\end{align}

Also, by doing the gradient with respect to $b$ and comparing with 0 we get:
\begin{align}
    \frac{\partial L}{\partial b} = - \sum \alpha_i y_i = 0 \implies \sum \alpha_i y_i = 0
\end{align}
 
If we plug these expressions back in equation \ref{eq:svm_lagrangian_from_primal} we can construct the \emph{dual form}. Note that $     ||\vt{w}||^2 = \vb{w}^\T \vb{w}$.

\begin{align*}
    L &= \frac{1}{2} \left( \sum \alpha_i y_i \vt{x_i} \right) \cdot \left( \sum \alpha_j y_j \vt{x_j} \right)
    - \sum \alpha_i \left[ y_i (\vb{w}^\T \vb{x_i} + b) - 1\right]
\\
    L &= \frac{1}{2} \left( \sum \alpha_i y_i \vt{x_i} \right) \cdot \left( \sum \alpha_j y_j \vt{x_j} \right)
    - \sum \alpha_i \left[ y_i \vb{w}^\T \vb{x_i} + y_i b - 1\right]
\\
    L &= \frac{1}{2} \left( \sum \alpha_i y_i \vt{x_i} \right) \cdot \left( \sum \alpha_j y_j \vt{x_j} \right)
    - \sum \left[ \alpha_i y_i \vb{w}^\T \vb{x_i} + \alpha_i y_i b - \alpha_i \right]
\\
    L &= \frac{1}{2} \left( \sum \alpha_i y_i \vt{x_i} \right) \cdot \left( \sum \alpha_j y_j \vt{x_j} \right)
    - \sum \alpha_i y_i \vb{w}^\T \vb{x_i} - \sum \alpha_i y_i b + \sum \alpha_i
\\
    L &= \frac{1}{2} \left( \sum \alpha_i y_i \vt{x_i} \right) \cdot \left( \sum \alpha_j y_j \vt{x_j} \right)
    - \sum \alpha_i y_i \vb{w}^\T \vb{x_i} - b \sum \alpha_i y_i + \sum \alpha_i 
\\
    L &= \frac{1}{2} \left( \sum \alpha_i y_i \vt{x_i} \right) \cdot \left( \sum \alpha_j y_j \vt{x_j} \right)
    - \sum \alpha_i y_i \vb{w}^\T \vb{x_i} + \sum \alpha_i 
\\
    L &= \frac{1}{2} \left( \sum \alpha_i y_i \vt{x_i} \right) \cdot \left( \sum \alpha_j y_j \vt{x_j} \right)
    - \vt{w} \cdot \sum \alpha_i y_i \vb{x_i} + \sum \alpha_i 
\\
    L &= \frac{1}{2} \left( \sum \alpha_i y_i \vt{x_i} \right) \cdot \left( \sum \alpha_j y_j \vt{x_j} \right)
    - \left( \sum \alpha_i y_i \vt{x_i} \right) \cdot \left( \sum \alpha_j y_j \vt{x_j} \right) + \sum \alpha_i 
\\
    L &= \sum \alpha_i - \frac{1}{2} \left( \sum \alpha_i y_i \vt{x_i} \right) \cdot \left( \sum \alpha_j y_j \vt{x_j} \right)
\end{align*}

\subsubsection*{Dual form}
\begin{align}
    \underset{\vt{\alpha}}{\text{arg} \max}\ \sum \alpha_i - \frac{1}{2} \sum \sum \alpha_i \alpha_j y_i y_j \vt{x_i} \cdot \vt{x_j}
    \text{\quad s.t. \quad} \forall \alpha_i : \alpha_i \ge 0 \land \sum \alpha_i y_i = 0
\end{align}

This variation of the optimization problem has the immediate advantage of op\-ti\-miz\-ing over $\vt{\alpha}$ instead of $\vt{w}$. In cases where the number of dimensions exceeds the number of examples, solving this formulation is more efficient. Another historical reason for which the dual form has been preferred is because it allows using kernels (\cite{chapelle_training_2007}), something we will see later in section \ref{sec:ch4.kernels}.

Both this form and the primal form will require the use of a quadratic op\-ti\-miza\-tion solver. The general complexity of such algorithms is $O(n^3)$. However, various techniques can be used that accomplish to reduce this complexity for both the primal and the dual forms.

For this dual formulation, instead of directly finding the values of $\vt{w}$ we find the values of $\vt{\alpha}$. If we want to later calculate the values of the weight vector we can use the \emph{representer theorem} (equation \ref{eq:svm_representer}). For the constant parameter $b$ we can use the margin boundary equation \ref{eq:gutters}, which with some algebra can be expressed as $b = y_i - \vb{w^\T}\vb{x_i}$.

\subsection{Regularization}

Until now, we've seen how it is desirable to maximize the margin, but we've pur\-pose\-ly ignored the common situation where examples are not linearly separable. It could simply be that classes follow a distribution that is not separable using a hyperplane, in which case we would use kernels. But it could also be the case that although classes are close to be linearly separable there is some overlapping in their distributions.  

We can solve this problem by redefining our margin as a \emph{soft margin}, such that we allow some amount of observations to be incorrectly classified, thus falling within the margin or in the wrong side of the decision boundary, but with a penalty that increases with the distance to correct side of the margin. To do so we introduce \emph{slack variables}, one for each example, such that $\xi_i = 0$ if the example $i$ is correctly classified, $0 < \xi_i < 1$ if within the margin, and $\xi_i \ge 1$ if it's in the wrong side of the margin.

Now we can reformulate our \emph{primal} with this new variables, penalizing the sum of misclassifications and also introducing a \emph{regularization parameter} $C$ that allows to specify the desired trade-off strength between correct classification and slack.

\subsubsection*{Primal form with regularization}
\begin{align}\label{eq:svm_primal_reg} 
    \underset{\vt{w}, b}{\text{arg} \min}\ \frac{1}{2} ||\vt{w}||^2 + C \sum_{i=1}^{N} \xi_i
    \text{\quad s.t. \quad} 
    \forall i : y_i (\vb{w}^\T \vb{x_i} + b) \ge 1 - \xi_i \quad \land \quad \xi_i \ge 0
\end{align}

An equivalent way to perform regularization is by using an empirical risk mini\-mization approx. Given our decision rule (equation \ref{eq:drule}), we need to find a \emph{loss function} that works well for classification problems, such as the \emph{hinge loss}, defined as:
\begin{align}
    l(t) = \max \{0, 1 - t\} \text{\quad where \quad} t = y_if(x_i) = y_i (\vb{w}^\T \vb{x_i} + b)
\end{align}

This can also be written:
\begin{align*}
    l(t) = \left\{
        \begin{array}{ll}
            0   & \mbox{if} \quad t \ge 1 \\
            1-t & \mbox{if} \quad t < 1  
        \end{array}
    \right.
\end{align*}

For a given training set we seek to minimize the total loss, while regularizing the objective with $l_2$-regularization (i.e. $||\vt{w}||^2$). This gives the unconstrained reg\-u\-lar\-iza\-tion problem equivalent to equation \ref{eq:svm_primal_reg}:
\begin{align}
    \underset{\vt{w}, b}{\text{arg} \min}\ \frac{1}{2} ||\vt{w}||^2 + C \sum_{i=1}^{N} \max \{0, 1 -  y_i (\vb{w}^\T \vb{x_i} + b)\}
\end{align}

Using the same Lagrangian process shown for the hard margin version, we can obtain the dual form of the soft margin version as follows:
\begin{align*}
    L(\vt{w}, b, \vt{\xi}, \vt{\alpha}, \vt{\gamma}) = 
    \frac{1}{2} ||\vt{w}||^2 + C \sum \xi_i
    - \sum \alpha_i \left[ y_i (\vb{w}^\T \vb{x_i} + b) - 1 + \xi_i \right]
    - \sum \gamma_i \xi_i
\end{align*}

\subsubsection*{Dual form with regularization}
\begin{align}\label{eq:svm_dual_reg} 
    \underset{\vt{\alpha}}{\text{arg} \max}\ \sum \alpha_i - \frac{1}{2} \sum \sum \alpha_i \alpha_j y_i y_j \vt{x_i} \cdot \vt{x_j}
    \text{\quad s.t. \quad} \forall \alpha_i : 0 \le \alpha_i \le C \land \sum \alpha_i y_i = 0
\end{align}

This is notably similar to the hard margin version, where only one constrain has changed. 

\pagebreak

\section{Kernel Methods}
\label{sec:ch4.kernels}

The use of kernel functions enables learning machines such as SVM to have non-linear decision boundaries, thus allowing them to make correct predictions even if the dataset is not linearly separable (but separable nonetheless), see Figure \ref{fig:ch4.kernels}.

\subsection{Feature map}

A feature map is a function $\phi : D \rightarrow H$ that transforms, or maps, or projects, vectors in some space $D$ (typically $\mathbb{R}^D$) to another space $H$. In the context of machine learning, $D$ is often the \emph{input space} (the space we've been working with until now) and $H$ the \emph{feature space}. When no feature map is used, there is no distinction between the two.

For the sake of using kernels, we restrict feature maps to those whose range $H$ is a \emph{Hilbert space}. A Hilbert space is a \emph{metric space} that defines an inner product and also is \emph{complete} with respect to the distance function induced by that inner product (\cite{noauthor_wikipedia_2021-1}).

To verify that a space is a real metric space we need to see that, for any vectors $\vb{x}$, $\vb{y}$, $\vb{z}$ and scalars $a$, $b$:

\begin{enumerate}
    \item The inner product is symmetric:
    \begin{align*}
        \ip{\vb{x}, \vb{y}} = \ip{\vb{x}, \vb{y}}
    \end{align*}
    \item The inner product is lineal:
    \begin{align*}
        \ip{a\vb{x} + b\vb{y}, \vb{z}} = a\ip{\vb{x}, \vb{z}} + b\ip{\vb{y}, \vb{z}} 
    \end{align*}
    \item The inner product of the same element is positive definite:
    \begin{align*}
        \ip{\vb{x}, \vb{x}} \ge 0
    \end{align*}
    Where $\ip{\vb{x}, \vb{x}} = 0$ only if $\vb{x}$ is neutral, i.e. $\vt{x} = (0,0,\dots, 0)$. \\
    These three properties are enough to define a \emph{product space}, to make it also a metric space we need to add the next two.
    \item The triangle inequality holds:
    \begin{align*}
        d(\vb{x}, \vb{z}) \le d(\vb{x}, \vb{y}) + d(\vb{y}, \vb{z}) 
    \end{align*}
    \item The more general Cauchy–Schwarz inequality, from which the triangle in\-equal\-i\-ty can actually be derived.
    \begin{align*}
        |\ip{\vb{x}, \vb{y}}| \le ||\vb{x}||\ ||\vb{y}|| 
    \end{align*}
\end{enumerate}

It is not hard to extend these properties to complex spaces, although outside the scope of this project. Finally, to see that this space is complete, we need to check that every Cauchy sequence in this space is convergent, i.e. has a limit also in the space. In other words, given a metric, such as the euclidean distance, there will always be points $\vb{x}$ and $\vb{y}$ such that for any $r > 0$ we have that $d(\vb{x}, \vb{y}) < r$. Euclidean spaces $\mathbb{R}^n$ as well as the complex space $\mathbb{C}$, and others, are examples of Hilbert spaces.

\subsection{Kernel functions}

A \emph{similarity function} is a function $d : \mathcal{X \times X} \rightarrow \mathbb{R}$ that uses some similarity measure (e.g. euclidean distance) to determine if two objects (e.g. points, vectors) are similar and returns a real number specifying how much. Kernel functions are a class of similarity functions for which a feature map is implicitly defined, such that:
\begin{align}
    k(\vb{x_i}, \vb{x_j}) = \ip{\phi(\vb{x_i}), \phi(\vb{x_j})}
\end{align}

We can compute explicitly a kernel function if we have the feature mapping defined, for instance, given:
\begin{align*}
    \phi : \mathbb{R}^2 \rightarrow \mathbb{R}^4 &
    \qquad\qquad
    \phi(x_1, x_2) = (x_1^2, x_2^2, x_1x_2, x_1x_2)
\end{align*}

Remember that the inner product in an euclidean space $\mathbb{R}^n$ is defined as the dot product. Then the kernel function, for two points $\vb{a}, \vb{b} \in \mathbb{R}^2$ is:
\begin{align*}
    k(\vb{a}, \vb{b}) &= \ip{\phi(\vb{a}), \phi(\vb{b})} = \ip{\phi(a_1, a_2), \phi(b_1, b_2)} \\
    &= a_1^2b_1^2 + a_2^2b_2^2 + a_1a_2b_1b_2 + a_1a_2b_1b_2 \\
    &= (a_1b_1)^2 + (a_2b_2)^2 + 2(a_1b_1)(a_2b_2) \\
    &= (a_1b_1 + a_2b_2)^2 = \ip{\vb{a}, \vb{b}}^2
\end{align*}

Notice how a kernel function may actually simplify the computation and reduce the required amount of operations compared to actually transforming the points and then applying the inner product. In this case, for in\-stance, we see that although the feature mapping is doubling the amount of di\-men\-sions, the kernel function that uses it can be simplified to an inner product in the domain. Thus, we can also define a kernel function without defining the feature map explicitly, e.g. $k(\vb{a}, \vb{b}) = \ip{\vb{a}, \vb{b}}^2$. Also note that there may be multiple feature maps that result in the same kernel. 

One powerful alternative technique to construct kernels implicitly is to build them out of simpler kernels as building blocks. This can be done using a set of known properties. Given valid kernels $k_1(\vb{a}, \vb{b})$ and $k_2(\vb{a}, \vb{b})$, with $\vb{a}, \vb{b} \in \mathcal{X}$, the following kernels are also valid:
\begin{align}
    k(\vb{a}, \vb{b}) &= c k_1(\vb{a}, \vb{b})\\
    k(\vb{a}, \vb{b}) &= f(\vb{a}) k_1(\vb{a}, \vb{b}) f(\vb{b})\\
    k(\vb{a}, \vb{b}) &= q(k_1(\vb{a}, \vb{b}))\\
    k(\vb{a}, \vb{b}) &= \text{exp}(k_1(\vb{a}, \vb{b}))\\
    k(\vb{a}, \vb{b}) &= k_1(\vb{a}, \vb{b}) + k_2(\vb{a}, \vb{b})\\
    k(\vb{a}, \vb{b}) &= k_1(\vb{a}, \vb{b}) k_2(\vb{a}, \vb{b})\\
    k(\vb{a}, \vb{b}) &= k_{r}(\phi(\vb{a}), \phi(\vb{b}))
\end{align} 
where $c > 0$ is a constant, $f(\cdot)$ is any function, $q(\cdot)$ is a polynomial function with nonnegative coefficients, and $k_r$ only applies to euclidean spaces $\mathbb{R}^n$ (\cite{bishop_pattern_2006}).

Using this knowledge we can construct some of the most popular kernels:

\subsubsection*{Linear kernel}
\begin{align}
    k(\vb{a}, \vb{b}) = \ip{\vb{a}, \vb{b}}
\end{align}

\subsubsection*{Polynomial kernel}
\begin{align}
    k(\vb{a}, \vb{b}) = (\ip{\vb{a}, \vb{b}} + c)^d
\end{align}
For some $d \in \mathbb{N}$ and $c \ge 0$, when $c = 0$ is said to be homogeneous.

\subsubsection*{RBF (Radial Basis Function) / Gaussian / Squared Exponential Kernel}
\begin{align}
    k(\vb{a}, \vb{b}) = \text{exp}(-\gamma ||\vb{a} - \vb{b}||^2)
\end{align}
Where $\gamma$ is a parameter that sets the “spread” of the kernel and $\text{exp}(x) = e^x$. Recall that a Gaussian distribution has a bell-shaped curve, that is, closer points have more similarity (and thus a greater value) than more separated points. By setting $\gamma = \frac{1}{2\sigma^2}$ we can write it in the equivalent Gaussian form:
\begin{align}
    k(\vb{a}, \vb{b}) = \text{exp}\left(-\frac{||\vb{a} - \vb{b}||^2}{2\sigma^2}\right)
\end{align}

This kernel has multiple interesting properties. It can be expressed as an infinite sum of polynomial kernels, this implies that the projection, i.e. the feature space, is a space with infinite dimension (\cite{bernstein_radial_2017}). In contrast with other kernels, where using the implicit kernel function only provides an improvement in com\-pu\-ta\-tion\-al cost, in this case not needing to calculate the feature map explicitly allows turning a problem that is not computable into one that can be computed trivially.

\begin{figure}[h]
    \centering
    \includegraphics[width=0.7\linewidth]{img/ch4/kernels.png}
    \caption{Input space decision boundary of SVM with different kernels. Source (\cite{deisenroth_mathematics_2020}).}
    \label{fig:ch4.kernels}
\end{figure}


\subsection{The kernel trick}

Recalling the SVM formulation with regularization, equation \ref{eq:svm_primal_reg}, what we want to do now is to use a feature map such that the SVM operates on the feature space instead of the input space. If we modify every instance $\vb{x}$ with $\phi(\vb{x})$, replace the dot product with the inner product, and produce the dual using the Lagrangian method over this new formulation, we will obtain:
\begin{align*}
    \underset{\vt{\alpha}}{\text{arg} \max}\ \sum \alpha_i - \frac{1}{2} \sum \sum \alpha_i \alpha_j y_i y_j \ip{\phi(\vt{x_i}), \phi(\vt{x_j})}
    \text{\quad s.t. \quad} \forall \alpha_i : 0 \le \alpha_i \le C \land \sum \alpha_i y_i = 0
\end{align*}

Notice that with this new formulation we can replace the inner product with a kernel function and profit from all the advantages that kernels provide over having to explicitly compute the inner product in the feature space. This is called the “kernel trick”.

\subsubsection*{SVM Dual form (kernel version)}
\begin{align}\label{eq:svm_dual_kernel} 
    \underset{\vt{\alpha}}{\text{arg} \max}\ \sum \alpha_i - \frac{1}{2} \sum \sum \alpha_i \alpha_j y_i y_j k(\vb{x_i}, \vb{x_j})
    \text{\ \ \ s.t. \ \ } \forall \alpha_i : 0 \le \alpha_i \le C \land \sum \alpha_i y_i = 0
\end{align}

With this new version we also have the possibility to precompute all values of the kernel. We do this by defining a matrix $K_{i,j} = k(\vb{x_i}, \vb{x_j})$ called \emph{Gram matrix} or sometimes \emph{kernel matrix}. Because of the properties of kernels, a kernel matrix will always be symmetric and positive semi-definite, i.e. $\forall \vb{z} \in \mathbb{R}^n : \vb{z}^T K \vb{z} \ge 0$.

In the context of image processing, a convolution between a kernel matrix (also called \emph{convolution matrix}) and an image is used to do all kinds of filtering, including blurring, sharpening, embossing, edge detection, and more. This specific kernel matrix is generated using all discrete points representing pixel positions (the center of cells in a grid) with an origin in the center of the grid.

Sometimes the dual SVM formulation will be expressed in terms of the kernel matrix directly. Another matrix formulation can be done by using the \emph{Hessian} matrix, which for SVM is defined as $\vb{H_{i,j}} = y_iy_jk(\vb{x_i}, \vb{x_j})$. This allows expressing equation \ref{eq:svm_dual_kernel} as a cost function (to be minimized) with the formula $(1/2) \boldsymbol{\alpha}^\T \vb{H} \boldsymbol{\alpha} - \alpha^\T \vb{1}$, where $\vb{1}$ (in bold) is a vector of 1.

Another advantage of using kernels is that, since they don't restrict the input space to real numbers, you can now use SVM to classify between all kind of objects, e.g. sets, sequences, strings, graphs and distributions.

\section{SVM-RFE}

Continuing our discussion on section \ref{sec:ch1.feature_selection}, SVM-RFE is an embedded feature se\-lec\-tion algorithm. In this section we shall explore in more detail how and why this algorithm works.

\subsection{Ranking criteria}

One basic idea of feature selection is to use a feature ranking and then select a number $r$ of the best ranking features as the final selection. For classification prob\-lems, one possible way to produce such a ranking is to evaluate how well a single feature contributes to the separation of the classes, these are called \emph{correlation filter methods} and the evaluation function \emph{correlation coefficients}, e.g:
\begin{align*}
    w_i = \frac{\mu_i(+) - \mu_i(-)}{\sigma_i(+) + \sigma_i(-)} 
\end{align*}
where $\mu_i$ and $\sigma_i$ are the mean and standard deviations for the observation in the $(+)$ and $(-)$ class respectively. Large positive values of $w_i$ indicate strong correlation with class $(+)$ and large negative values with class $(-)$. The absolute or the square of $w_i$ can be used as the \emph{ranking criteria}, i.e. the scores of each feature such that when sorted produce a feature ranking. Other similar correlation based ranking criterion include \emph{Fisher}, \emph{Pearson}, \emph{Spearman} or \emph{Kendall} coefficients.

Note that a feature that is correlated to some class is by itself a class predictor (albeit an imperfect one). This class predictors may be combined, either directly or with some scheme, e.g. \emph{weighted voting}, to produce a linear discriminant classifier (\cite{guyon_gene_2002}).
\begin{align*}
    D(\vt{x}) = \vt{w} \cdot (\vt{x} - \mu) 
\end{align*}

The opposite is also true, given a wight vector, e.g. one produced by an SVM, the individual coordinates of the vector can be interpreted as individual correlation coefficients. The inputs that are weighted by the largest value influence most the classification decision. Therefore, if the classifier performs well, those inputs with the largest weights correspond to the most informative features.

An alternative geometric interpretation, which leads to the same result, can also be made. By definition, the wight vector produced by a linear SVM corresponds to the normal vector of the decision boundary (a hyperplane). Setting one of the coordinates of such vector $\vt{w_i}$ to $0$ will change its direction and thus produce a rotation on the hyperplane. This rotation will modify the decision regions, thus, points located in the space where the region changes (the intersection space) will be classified incorrectly (assuming no regularization). If we want to reduce the classification error a good idea is to reduce the space in the intersection, which is equivalent to reducing the amount of rotation on the hyperplane or the change in direction of the normal vector. Coordinates with a value close to $0$, when the value is set to $0$, will produce small changes in direction, whilst coordinates where the difference $|0 - w_i|$ is bigger will produce bigger changes. I.e, coordinates with the largest weights correspond to the most informative features (when they are removed more error is produced) and coordinates with the lowest weight are the less informative. As such, we can make a ranking criterion by simply taking the absolute value $|w_i|$, or the square $(w_i)^2$, of the coordinates of the wight vector produced by an SVM.

One problem of this interpretation is that it only works when the decision bound\-ary is lineal. However, it gives us a hint on how to make a generalization for non-linear decision boundaries: We can use the change in objective function (or cost function) when a feature is removed as a ranking criterion. Let $J$ be the SVM cost function $J = (1/2) \boldsymbol{\alpha}^\T \vb{H} \boldsymbol{\alpha} - \alpha^\T \vb{1}$. First, we train the machine and compute the $\boldsymbol{\alpha}$ parameters. Then, to compute the change in cost function $DJ(i)$ caused by removing the feature $i$, we leave the $\boldsymbol{\alpha}$ unchanged (for performance reasons) and recompute the matrix $\vb{H}$, yielding $\vb{H}(-i)$ where the notation $(-i)$ means that the component $i$ has been removed. The resulting ranking coefficient is:
\begin{align*}
    DJ(i) = [(1/2) \boldsymbol{\alpha}^\T \vb{H} \boldsymbol{\alpha} - \alpha^\T \vb{1}]
     - [(1/2) \boldsymbol{\alpha}^\T \vb{H}(-i) \boldsymbol{\alpha} - \alpha^\T \vb{1}]
\end{align*}

\subsubsection*{General ranking coefficient for SVM}
\begin{align}
    DJ(i) = (1/2)(\boldsymbol{\alpha}^\T \vb{H} \boldsymbol{\alpha} - \boldsymbol{\alpha}^\T \vb{H}(-i) \boldsymbol{\alpha})
\end{align}

Note that when the kernel is lineal then $\boldsymbol{\alpha}^\T \vb{H} \boldsymbol{\alpha} = ||\vb{w}||^2$, therefore:
\begin{align*}
    DJ(i) &= (1/2)(||\vb{w}||^2 - ||\vb{w}(-i)||^2) \\
          &= (1/2)[(w_1^2 + w_2^2 + \dots + w_D^2) - (w_1^2 + \dots + w_{i-1}^2 + 0 + w_{i+1}^2 + \dots + w_D^2)] \\
          &= (1/2)(w_i^2) \quad \sim \quad (w_i)^2
\end{align*} 

Computationally, the non-lineal version is more expensive, however we may be able to optimize it by only recomputing support vectors (since $\boldsymbol{\alpha}$ is unchanged), and by caching partial results on the other components of these vectors.

\begin{sloppypar}
There are other ranking coefficients that use the information form the model trained by some learning machine, e.g. \emph{Relevancy and Redundancy Criteria} (\cite{mundra_svm-rfe_2007}), or the \emph{Span Estimate Gradient} (\cite{rakotomamonjy_variable_2003}).
\end{sloppypar}

Particularly, one method that also returns a ranking coefficient $(w_i)^2$ for the lineal case is based on the OBD algorithm (\cite{guyon_gene_2002}). It approximates the dif\-fer\-ence with a Taylor series expansion\footnote{Any real or complex differentiable function can be expressed as a Taylor series, which is a polynomial of the form $f(a) + \frac{f'(a)}{1!}(x - a) + \frac{f''(a)}{2!}(x - a)^2 + \dots$, where $a$ is some point in the domain. The function can thus be approximated by restricting the polynomial to some order $n$.} to second order. At the optimum of $J$ the first order term can be neglected, yielding:
\begin{align*}
    DJ(i) = (1/2) \frac{\partial^2 J}{\partial w_i^2} (0 - w_i)^2 = \frac{\partial \vb{w}}{\partial w_i} (w_i)^2 = (w_i)^2
\end{align*}

\subsection{Recursive Feature Elimination}

The criteria $DJ(i)$ estimates the effect of removing one feature at a time on the objective function. It doesn't perform particularly well when several features are removed. This problem can be overcome by using an iterative procedure, called RFE (Recursive Feature Elimination), where we eliminate one feature and retrain the learning machine each iteration.

This is an instance of backward feature elimination, and since we're using the wight vector of the learning machine in order to compute the feature ranking, we're performing an embedded method. For performance reasons it may be more efficient to remove multiple features at a time, thus we can introduce a \emph{step} parameter $t$ to indicate how many features to do at a time. In this case the method produces a feature subset ranking, a nested set of the form $F_1 \subset F_2 \subset \dots \subset F$. This feature subset ranking may better be represented as a list of vectors, such that when flat\-tened produces the final feature ranking.

Note that with this method, top ranked features are not necessarily the most relevant when taken individually, it is when taken together that features of a subset $F_m$ are relevant in some sense.

In the following figure we present a pseudocode implementation of the SVM-RFE algorithm, which is the particular case of RFE where SVM are used. For clarity purposes this implementation is restricted to linear SVM and the associated ranking criteria.

\begin{algorithm}[h]
    \DontPrintSemicolon
      \KwInput{$t$ \tcp*{t = step}}
      \KwOutput{$\vt{r}$}
      \KwData{$X_0,\vt{y}$}
      $\vt{s} = [1,2, \dotsc, n]$ \tcp*{subset of surviving features}
      $\vt{r} = []$ \tcp*{feature ranked list} 
      \While{$|\vt{s}| > 0$}
        {
            \tcc*[h]{Restrict training examples to good feature indices}\\
            $X=X_0(:,\vt{s})$\VS

            \tcc*[h]{Train the classifier}\\
            $\vt{\alpha} = \texttt{SVM-train(} X, y \texttt{)}$\VS

            \tcc*[h]{Compute the weight vector of dimension length $|\vt{s}|$}\\
            $\vt{w} = \sum_k{\vt{\alpha_k} \vt{y_k} \vt{x_k}}$\VS

            \tcc*[h]{Compute the ranking criteria}\\
            $\vt{c} = [(w_i)^2 \text{ for all $i$}]$\VS

            \tcc*[h]{Find the $t$ features with the smallest ranking criterion}\\
            $\vt{f} = \texttt{argsort}(\vt{c})(\ :t)$\VS

            \tcc*[h]{Update the feature ranking list}\\
            $\vt{r} = [\vt{s}(\vt{f}), ...\vt{r}]$\VS

            \tcc*[h]{Eliminate the features with the $t$ smallest ranking criterion}\\
            $\vt{s} = [[...\vt{s}(1:f_i - 1), ...\vt{s}(f_i + 1:|\vt{s}|)]$ for all $i]$
        }
    \caption{SVM-RFE}
\end{algorithm}

Note that, although we're using vectors to store the feature ranking, by using another structure (like a dictionary in \texttt{Python}) we could store the specific feature subsets, this is actually done in some implementations. Also, this code doesn't perform any kind caching, nor does it test the machine with a validation set so that an accuracy score can be produced.

\subsection{Assessing performance}

Since learning machines will produce different levels of performance depending on the dataset, a baseline feature selection method is required in order to make proper comparisons. We can use a \emph{random feature selection} as the baseline method. This method is a good baseline because we expect that no method should perform worse than a purely uninformed one. It simply consists in making a random feature ranking and testing the accuracy of SVM with multiple feature subsets, specifically we make each feature subset by removing $t$ features each time based on the ranking order. For numerical stability we may also use cross-validation and take the mean of the results. Note that, although with this method the feature ranking is made at no cost, evaluating the accuracy at each step is just as computationally expensive as SVM-RFE itself, since one SVM training has to be performed each iteration. It is expected that this method will return acceptable results when all features are used, and decrease linearly in accuracy as features are removed.

When using SVM-RFE we actually have two different methods for evaluating the performance of the selection. As an embedded method, we can evaluate the accuracy at each iteration of the algorithm, by first passing a validation set to it. If the feature subsets evaluated are of the same size as in the random selection, i.e. the same step is used, this will make a fair comparison. Another option is to use SVM-RFE purely for generating the ranking and, after that, take the same approx we've used for random selection, fit and test at each iteration. This second approx will be as Computationally expensive as running SVM-RFE twice, but it allows producing performance metrics at the same intervals as the baseline method, even when the SVM-RFE algorithm uses different ones, thus making a fair comparison. In this case SVM-RFE is actually acting like a wrapper method instead of an embedded one.

The following comparison (Figure \ref{fig:ch4.dynamicStep.vanilla.comp}) has been made with an artificial dataset of 300 features, only 50 of which informative. Note that we provide both train and test accuracy, only the second is a valid metric of the expected accuracy on new samples, however, plotting the train accuracy is also useful to determine the overfitting as well as provide clues on the behavior of the selection algorithm.

\begin{figure}[H]
    \centering
    \begin{subfigure}[b]{0.4\linewidth}
        \includegraphics[width=\linewidth]{img/ch4/vanilla300-random.png}
        \subcaption*{\textbf{A.} Random selection}
    \end{subfigure}
    \begin{subfigure}[b]{0.4\linewidth}
        \includegraphics[width=\linewidth]{img/ch4/vanilla300-svmrfe-s300.png}
        \subcaption*{\textbf{B.} SVM-based filter method, step = 300}
    \end{subfigure}
    \begin{subfigure}[b]{0.4\linewidth}
        \includegraphics[width=\linewidth]{img/ch4/vanilla300-svmrfe-s10.png}
        \subcaption*{\textbf{C.} SVM-RFE, step = 10}
    \end{subfigure}
    \begin{subfigure}[b]{0.4\linewidth}
        \includegraphics[width=\linewidth]{img/ch4/vanilla300-svmrfe.png}
        \subcaption*{\textbf{D.} SVM-RFE, step = 1}
    \end{subfigure}
    \caption{Mean accuracy using 20-fold CV of an SVM classifier with different selection methods.}
    \label{fig:ch4.dynamicStep.vanilla.comp}
\end{figure}

As expected, random selection (plot \textbf{A}) performs significatively worse than any other method. We also verify, by comparing plots \textbf{B} and \textbf{C}, our previous assumption that the more iterations (smaller step) the better the accuracy. This relationship, however, doesn't seem to be lineal, as the same effect can not be appreciated when comparing plots \textbf{C} and \textbf{D}.

Notice that when evaluating the performance of a feature selection algorithm two variables must be considered, one is accuracy performance and the other is amount of features. A selection with a great accuracy but a lot of features may not be as good as a section with a decent accuracy and only a handful of features. That is, we want to maximize accuracy while minimizing amount of features. These two variables do not operate on the same range, and it can even be the case that the importance of one over the other is problem-dependent. This makes comparing multiple instances hard, this is why we are comparing pairs of plots instead of simply a pair of values.

Formally this is known as a multi-objective optimization problem. In this kind of problems a solution is not necessary unique, instead it is a set of \emph{Pareto optimal}\footnote{A concept more commonly used in game theory, decision theory or economics.} solutions, i.e. solutions such that no other solution in the set of feasible so\-lu\-tions is better. Of course this also means that all Pareto optimal solutions are just as good. To score the solutions a \emph{scalarization function} is used, such that the multi-objective problem becomes a single objective optimization problem. This function establishes a degree of trade-off between the different objectives and must be chosen based on expert criteria, i.e. the user decides what is more important for the specific problem at hand.

If the criteria is known beforehand, it can be used to guide the optimization problem. In our case, for example, it could be used to guide model selection, see (Section \ref{sec:ch4.performance}). This is known as an a priori method. If all feasible solutions (or a rep\-re\-sen\-ta\-tive subset) are first computed, this is known as a posteriori method. One typical scalar\-iza\-tion function is the weighted sum, also called \emph{linear scalarization}.

\subsubsection*{Linear Scalarization}
\begin{align} 
    \underset{\vb{x} \in \vb{X}}{\text{arg} \min}\ \sum_{i=1}^k w_i f_i(\vb{x})
\end{align}
where each $w_i$ is a weight and each $f_i(\vb{x})$ is a cost function (\cite{vasumathi_scalarizing_2019}). In our case we would only have two objective functions, the accuracy at each feature subset $\texttt{Acc}(F_i)$, and the percentage of features selected $|F_i| / |F|$. Thus, the scalar\-iza\-tion function will be of the form:
\begin{align*} 
    \underset{F_i \in F}{\text{arg} \min}\ w_1 (1 - \texttt{Acc}(F_i)) + w_2 (|F_i| / |F|)
\end{align*}

Using the wrapper form of SVM-RFE we can calculate the optimum on the val\-i\-da\-tion set and plot it. This is an a posteriori method.

\begin{figure}[H]
    \centering
    \begin{subfigure}[b]{0.4\linewidth}
        \includegraphics[width=\linewidth]{img/ch4/vanilla200-tradeoff.png}
        \subcaption*{$w = (0.95, 0.05)$}
    \end{subfigure}
    \begin{subfigure}[b]{0.4\linewidth}
        \includegraphics[width=\linewidth]{img/ch4/vanilla200-tradeoffB.png}
        \subcaption*{$w = (0.6, 0.4)$}
    \end{subfigure}
    \caption{Different optimums caused by changing the criteria for the linear scalarization function, projected as a red line.}
    \label{fig:ch4.tradeoff}
\end{figure}
\chapter{Experiments}
\label{Chapter5}

This chapter is organized in sections corresponding to each of the extensions we've outlined in section \ref{sec:ch1.objective}. For each extension we provide a general description of the idea with the rationale behind it, a brief complexity analysis, a pseudocode, and a detailed analysis of the experimental results.

The first section of this chapter does not correspond to any extension and is instead an introductory block.

\section{Introduction}

\subsection{General Framework}

\texttt{Sklearn}, being a general machine learning library, has its own implementation of the base \texttt{RFE} algorithm. Given that it is open source, we were able to base our own implementation on it. After pruning the code to its bare minimum (no dep\-en\-den\-cies), we began to add our own extensions. We organized each extension in its own folder, isolated from the rest, and copied the base implementation there.

For the actual experiments we've used \texttt{Jupiter Notebooks} (integrated in \texttt{Visual Studio Code}). Each experiment may use different notebooks depending on what is being tested exactly. Some code that is common to all notebooks, such a code for plotting, has been placed at the beginning (for convenience). The implementation of SVM-RFE and associated usability methods has been moved to separate Python files. This is both a convenient software pattern and a requirement of the parallelization library on Windows.

We use \emph{k-fold cross-validation} to perform the experiments multiple times with different folds, as explained in section \ref{sec:ch4.performance}. For the same experiment, the same amount of folds is used, but different amounts may be used in different experiments. We've parallelized this procedure with the standard \texttt{multiprocessing.pool} library. Given that our CPU has 8 cores, for the most computationally expensive experiments we're using 6 or 7-fold cross-validation in order to be able to finish the experiment in a single round (with one or two spare cores to be able to keep working in the machine). The execution time is always calculated as the mean of the elapsed times in every SVM-RFE execution (single core).

A limitation of Python parallelization when using Jupiter is that standard output gets suppressed. In order to debug (since debugging tools also don't work in this context) we raised exceptions, which are not suppressed because they reach the main thread. In some situations we've used other methods such a file logging or temporally running the code without parallelization.

For plotting we've used the \texttt{Mathplotlib} library. We have made a custom plot function that includes all relevant information for SVM-RFE. Having a standard plot design allows for easy comparison as shown in Figure \ref{fig:ch4.tradeoff}. This plot displays the accuracy (vertical axis) of each feature subset (by size, horizontal axis) selected by the SVM-RFE algorithm. We calculate and display both the train accuracy and the test accuracy (which is that of the va\-lid\-ation set when cross-validation is used). We also show where the optimal is based on the linear scalarization method. Furthermore, we also show what the minimum accuracy is, based on the amount of classes, with a doted red horizontal line. And finally we show with an overlay scatter plot of small vertical lines what the actual feature selection subset sizes where and what accuracy they had during the execution of SVM-RFE (training accuracy). It is important not to confuse this plot with plots describing an iterative optimization process or model selection.

\subsection{The data}

\subsubsection*{Sklearn Generator}

The general machine learning library \texttt{Sklearn} provides tools to generate artificial datasets for various prob\-lems. We're using the \texttt{make\_classification} gen\-er\-a\-tor, which creates normally-distributed clus\-ters of points placed at the vertices of an hypercube. Multiple clusters can corresponding to a single class. This specific gen\-er\-a\-tor also introduces in\-ter\-de\-pen\-dence between features and var\-i\-ous types of noise.

We've selected this generator because it allows to specify the amount of in\-for\-ma\-tive (i.e. non-redundant, useful) features the dataset should have, it can generate multi-class datasets, and by changing the amount of clusters per class it can control the separability of the data.

We don't use a single dataset of this kind in our experiments. Instead, we use datasets generated with the parameters that we think can better help evaluate the expected properties and correctness of the extensions.

\subsubsection*{MADELON}

This is one of the five datasets proposed for the NIPS (Neural Information Processing Systems) 2003 challenge in feature selection. The dataset remains publicly available in the UCI (University of California, Irvine) machine learning repository. The results of this challenge can be found at the workshop web page (\cite{guyon_result_2004}).

The winner of the challenge got an accuracy of 92.89\% with 8 selected features. However, this is using test data that is not publicly available. Using only the avail\-able data we've found articles describing the use of this dataset where the maximum accuracy reached is 88\%. 

This dataset is constructed similarly to the \texttt{sklearn} \texttt{make\_classification} gen\-er\-a\-tor. It uses clusters of points placed at the vertices of some hypercube, however, instead of a single hypercube it uses five of them. Also, this method labels each in hypercube in one of two classes randomly. Five features correspond to the which hypercube a point is in, with an extra 15 being redundant features extracted from linear combinations of the first 5. The total amount of features per data point is 500, 480 of them being noise (also called probes).

Of this dataset, 2000 observations are publicly available and where used in this project, 600 are part of a separate validation set not used in this project, and 1800 are part of a test set not used in this project and also not publicly available. All sets have the same amount of positive and negative samples.

\subsubsection*{Digits}

For the multi-class classification extension we've used the digits dataset (Optical Recognition of Handwritten Digits Data Set). This is also a dataset available in the UCI machine learning repository and is also provided ready to use in \texttt{Sklearn}. This dataset contains 5620 instances of 64 features corresponding to 10 possible handwritten digits. Each feature has been extracted from a 32×32 bitmap by count\-ing the pixels of 4×4 non-overlapping blocks. There are 64 informative features.

For this dataset common accuracy results are around 98\%, but we've found some cases where it can reach 100\%.

\begin{table}
    \centering
    \begin{tabular}{l c c c c}
    \toprule
    \tabhead{Name}      & \tabhead{Observations} & \tabhead{Features} & \tabhead{Informative}& \tabhead{Classes} \\
    \midrule
    \texttt{make\_classification}   & - & - & - & - \\
    MADELON                         & 2000 & 500 & 5 & 2 \\
    Digits                          & 1797 & 64 & ?? & 10 \\
    \bottomrule\\
    \end{tabular}
    \caption{Summary of dataset properties.}
    \label{tab:ch5.datasetdesc}
\end{table}

% -----------------------------------------------------------------------------------------

\section{Dynamic Step}

This extension is based on the constant step variant of SVM-RFE (Algorithm \ref{alg:rfe1}), however, instead of using some constant number $t$ as the step in each iteration, we calculate that number dynamically. The most straightforward way to do this is by using a percentage.

\subsection{Description and reasoning}
\label{sec:dynamicStep.desc}

The percentage is a hyper-parameter. It is used within every iteration to eliminate a number of the least ranked features. A constant step has already been used in pract\-ice, but it is expected that this method will be significantly faster without effecting the accuracy performance, or even improving it.

Other similar modifications are also found in the literature, including using the square root of the remaining features \texttt{SQRT-RFE}, an entropy based function \texttt{E-RFE}, or \texttt{RFE-Annealing} which sets the step at $|\vt{s}| \frac{1}{i+1}$, thus, changing the percentage each iteration (\cite{ding_improving_2006}).

\iffalse
That is, the amount of features eliminated "$r$" at $j$-th iteration is the summation of the amount of features eliminated each iteration, i.e. $np^i$:

\begin{equation}\label{eq:dynamicStep1}
    r = \sum_{i = 1}^{j}{(np^i)} 
\end{equation}

It may be more interesting to see complexity as in the amount of iterations re\-quired to eliminate $r$ features, it can be derived from equation \ref{eq:dynamicStep1} as follows:

\begin{align*}
    r &= n \sum_{i = 1}^{j}{p^i} = n \sum_{i = 0}^{j-1}{(p^i)} - n + np^j \\
    \frac{r}{n} &= \frac{1-p^j}{1-p} - 1 + p^j \\
    \frac{r}{n} &= \frac{(1-p^j) - (1-p) + (1-p)p^j}{1-p}\\
    \frac{(1-p)r}{n} &= (1-p^j) - (1-p) + (p^j-p^{j+1})\\
    \frac{(1-p)r}{n} &= 1 -p^j -1 + p + p^j - p^{j+1}\\
    \frac{(1-p)r}{n} &= p - p^{j+1}\\
    - \frac{(1-p)r}{n} + p &= p^{j+1}\\
    \log_{p} \left( - \frac{(1-p)r - np}{n} \right) &= \log_{p} (p^{j+1})\\
    \log_{p} \left( - \frac{(1-p)r - np}{n} \right) &= j + 1\\
    \log_{1/p} \left( - \frac{n}{(1-p)r} + 1/p \right) &= j + 1\\
\end{align*}
\fi

We assume that, the bigger the step each iteration, the worse the performance of the ranking. This is consistent with what we've seen in figure \ref{fig:ch4.dynamicStep.vanilla.comp}. However, since we're eliminating the worst variables first, eliminating more of them at once shouldn't affect performance because it is likely they would've been eliminated in the next iterations anyway. The fewer the iterations remaining, the riskier it becomes to eliminate multiple variables at once, and thus a smaller step is beneficial.

\subsubsection*{Our expectations for this extension are:}

\begin{itemize}
    \item \textbf{Improvement in time complexity:} Given that SVM complexity $O(dn^2)$ is lin\-ear\-ly dependent on the amount of dimensions $d$, we know that each iteration is faster than the one before it. By increasing the step in the first iterations we should drastically reduce the time it takes to complete, even if then we decide to reduce the step in later, less time-consuming, iterations.
    \item \textbf{Improvement in accuracy performance:} Using dynamic step, compared to a constant step $t > 1$, can also improve the accuracy. This is because the last iterations may be performed with a step $t \ge i \ge 1$.
\end{itemize}

Note that by adjusting the percentage you can decide for which of these two objectives you want to optimize, it may also be possible to find a middle case in which both the accuracy and the time are improved.

\subsection{Pseudocode formalization}

\begin{algorithm}[H]
    \DontPrintSemicolon
      \KwInput{$p$ \tcp*{p = percentage, $0 \le p \le 1$}}
      \KwOutput{$\vt{r}$}
      \KwData{$X_0,\vt{y}$}
      $\vt{s} = [1,2, \dotsc, d]$ \tcp*{subset of surviving features}
      $\vt{r} = []$ \tcp*{feature ranking list} 
      \While{$|\vt{s}| > 0$}
        {
            \tcc*[h]{Restrict training examples to good feature indices}\\
            $X=X_0(:,\vt{s})$\VS

            \tcc*[h]{Train the classifier}\\
            $\vt{\alpha} = \texttt{SVM-train(} X, y \texttt{)}$\VS

            \tcc*[h]{Compute the weight vector of dimension length $|\vt{s}|$}\\
            $\vt{w} = \sum_k{\vt{\alpha_k} \vt{y_k} \vt{x_k}}$\VS

            \tcc*[h]{Compute the ranking criteria}\\
            $\vt{c} = [(w_i)^2 \text{ for all $i$}]$\VS

            \tcc*[h]{Compute $t$ based on the percentage}\\
            $t = p|\vt{s}|$\VS

            \tcc*[h]{Find the $t$ features with the smallest ranking criterion}\\
            $\vt{f} = \texttt{argsort}(\vt{c})(\ :t)$\VS

            \tcc*[h]{Iterate over the feature subset}\\
            \For{$f_i \in \vt{f}$}{
                \tcc*[h]{Update the feature ranking list}\\
                $\vt{r} = [\vt{s}(f_i), ...\vt{r}]$\VS
    
                \tcc*[h]{Eliminate the feature selected}\\
                $\vt{s} = [...\vt{s}(1:f_i - 1), ...\vt{s}(f_i + 1:|\vt{s}|)]$
            }
        }
    \caption{SVM-RFE with DynamicStep}
\end{algorithm}
\VS
Note that since we do a non-lineal skip focused on the iterations that take more time we can achieve a reduced complexity of $O(\log(d)dn^2)$. This can be illustrated in Figure \ref{fig:ch5.dstep.comparetime}, where we show the relationship between the amount of remaining features and iterations for the different step methods. 

\begin{figure}[h]
    \centering
    \includegraphics[width=0.4\linewidth]{img/ch5/comparetimes.png}
    \caption{Amount of features remaining (vertical axis) at each iteration (horizontal axis).}
    \label{fig:ch5.dstep.comparetime}
\end{figure}

Note how the relative time cost of each method can be estimated as the area under the curve.

\subsection{Results}

\subsubsection*{Analysis with artificially generated data}

We generate a 2 class dataset with the following code. The scalarization trade-off used is of 80\% accuracy, 20\% feature subset size. All results are mean values extracted from a 7-fold cross-validation procedure.

\begin{verbatim}
    X, y = make_classification(
        n_samples = 1000, n_clusters_per_class=3, n_features = 300,
        n_informative = 100, n_redundant=100, n_repeated=20,
        flip_y=0.05, random_state=2, class_sep=2
    )
\end{verbatim}

We start by comparing how the dataset performs under a random feature se\-lec\-tion or a simple filter method. For the validation phase a linear SVM has been used, with the regularization parameter found by grid search and being $C = 0.00001$.

\begin{figure}[H]
    \centering
    \begin{subfigure}[b]{0.4\linewidth}
        \includegraphics[width=\linewidth]{img/ch5/dstep/random.png}
        \subcaption*{Random AT (85, 0.83, 0.187)}
    \end{subfigure}
    \begin{subfigure}[b]{0.4\linewidth}
        \includegraphics[width=\linewidth]{img/ch5/dstep/filter.png}
        \subcaption*{Filter AT (42, 0.88, 0.122)}
    \end{subfigure}
    \caption{Accuracy of feature rankings produced either by a random method or an SVM-based filter method. AT (\emph{feat.}, \emph{acc.}, \emph{cost})}
    \label{fig:ch5.dstep.init}
\end{figure}

Based on these results we expect that SVM-RFE must perform better than the filter method. Or objective now is to find if using a dynamic step can perform similarly or better than SVM-RFE in less amount of time. From a grid search model selection procedure we'be selected the best feature rankings, shown in Figure \ref{fig:ch5.dstep.vanillabest}.

\begin{figure}[H]
    \centering
    \begin{subfigure}[b]{0.4\linewidth}
        \includegraphics[width=\linewidth]{img/ch5/dstep/vanilla1.png}
        \subcaption*{Const Step $t=2, C=0.00001$}
    \end{subfigure}
    \begin{subfigure}[b]{0.4\linewidth}
        \includegraphics[width=\linewidth]{img/ch5/dstep/vanilla2.png}
        \subcaption*{Dynamic Step $p=0.05, C=0.0001$}
    \end{subfigure}
    \caption{Accuracy of the best feature rankings produced by SVM-RFE with constant or dynamic step.}
    \label{fig:ch5.dstep.vanillabest}
\end{figure}

In the following tables we detail the results of the model selection. The cell corresponding to the best model of each algorithm (by cost) and associated time have been highlighted. 
\begin{table}[H]
    \centering
    \begin{tabular}{l | c c c|c c c|c c c}
        \toprule
        \multicolumn{1}{c}{Step} & \multicolumn{3}{c}{\textbf{2}} & \multicolumn{3}{c}{\textbf{10}} & \multicolumn{3}{c}{\textbf{50}}\\
        %\cline{2-4}\cline{5-7}\cline{8-10}
        \midrule
        \textbf{$C$}&Feat.&Acc.&Cost&Feat.&Acc.&Cost&Feat.&Acc.&Cost \\
        \midrule
        \textbf{0.000001} &    39 & 88.40\% & 0.119 &    41 & 89.40\% & 0.112 &    38 & 88.70\% & 0.116\\
        \textbf{0.000010} &    \mrk{39} & \mrk{89.90\%} & \mrk{0.107} &    57 & 90.70\% & 0.112 &    54 & 89.80\% & 0.118\\
        \textbf{0.000100} &    55 & 91.00\% & 0.109 &    61 & 91.30\% & 0.110 &    59 & 90.10\% & 0.118\\
        \textbf{0.001000} &    81 & 89.40\% & 0.139 &    77 & 87.90\% & 0.148 &    81 & 87.10\% & 0.157\\
        \textbf{0.010000} &   104 & 88.70\% & 0.160 &   103 & 89.10\% & 0.156 &   116 & 87.90\% & 0.174\\
        \bottomrule
        \end{tabular}
    \caption{Grid search of SVM-RFE with constant step.}
\end{table}

\begin{table}[H]
    \centering
    \begin{tabular}{l | c c c}
        \toprule
        \multicolumn{1}{c}{\textbf{C/Step}} & \textbf{2} & \textbf{10} & \textbf{50} \\
        %\cline{2-4}\cline{5-7}\cline{8-10}
        \midrule
        \textbf{0.000001} & 0:03.691 & 0:00.643 & 0:00.124\\
        \textbf{0.000010} & \mrk{0:05.242} & 0:01.050 & 0:00.190\\
        \textbf{0.000100} & 0:07.001 & 0:01.440 & 0:00.279\\
        \textbf{0.001000} & 0:08.039 & 0:01.522 & 0:00.334\\
        \textbf{0.010000} & 0:11.834 & 0:02.424 & 0:00.577\\
        \bottomrule
        \end{tabular}
    \caption{Execution time (min:sec.msec) of SVM-RFE with constant step.}
\end{table}

\begin{table}[H]
    \centering
    \begin{tabular}{l | c c c|c c c|c c c}
        \toprule
        \multicolumn{1}{c}{Percentage} & \multicolumn{3}{c}{\textbf{0.04}} & \multicolumn{3}{c}{\textbf{0.12}} & \multicolumn{3}{c}{\textbf{0.20}}\\
        %\cline{2-4}\cline{5-7}\cline{8-10}
        \midrule
        \textbf{$C$}&Feat.&Acc.&Cost&Feat.&Acc.&Cost&Feat.&Acc.&Cost \\
        \midrule
        \textbf{0.000001} &    26 & 88.20\% & 0.112 &    22 & 86.60\% & 0.122 &    37 & 88.10\% & 0.120\\
        \textbf{0.000010} &    43 & 89.80\% & 0.110 &    36 & 89.80\% & 0.106 &    60 & 89.30\% & 0.126\\
        \textbf{0.000100} &    \mrk{34} & \mrk{90.00\%} & \mrk{0.103} &    59 & 91.00\% & 0.111 &    42 & 90.40\% & 0.105\\
        \textbf{0.001000} &    56 & 87.30\% & 0.139 &    63 & 87.20\% & 0.144 &    91 & 89.00\% & 0.149\\
        \textbf{0.010000} &    87 & 88.40\% & 0.151 &   104 & 87.40\% & 0.170 &   107 & 87.90\% & 0.168\\
        \bottomrule
        \end{tabular}
    \caption{Grid search of SVM-RFE with dynamic step.}
\end{table}

\begin{table}[H]
    \centering
    \begin{tabular}{l | c c c}
        \toprule
        \multicolumn{1}{c}{\textbf{C/Percentage}} & \textbf{0.04} & \textbf{0.12} & \textbf{0.20} \\
        %\cline{2-4}\cline{5-7}\cline{8-10}
        \midrule
        \textbf{0.000001} & 0:01.048 & 0:00.324 & 0:00.202\\
        \textbf{0.000010} & 0:01.455 & 0:00.443 & 0:00.283\\
        \textbf{0.000100} & \mrk{0:01.947} & 0:00.612 & 0:00.380\\
        \textbf{0.001000} & 0:02.505 & 0:00.709 & 0:00.404\\
        \textbf{0.010000} & 0:03.487 & 0:01.079 & 0:00.639\\
        \bottomrule
        \end{tabular}
    \caption{Execution time (min:sec.msec) of SVM-RFE with dynamic step.}
\end{table}

Note that the three best models for dynamic step are all better than the best model using constant step. This results however have to be taken with a grain of salt, given that variance is present and the difference is only marginal.

A more clear-cut improvement can be seen on the computational cost, which had a speedup of x2.7 even when the model used by dynamic step had a greater value of $C$. (Equation \ref{eq:ch5.dstep.speedup1}).
\label{eq:ch5.dstep.speedup1}
\begin{align}
    \text{Speedup} = \frac{T_{\text{old}}}{T_{\text{new}}} = \frac{5.242}{1.947} = 2.692
\end{align}


\subsubsection*{Initial Analysis with Madelon}

We've initially run some tests to find how hard it is for the SVM-RFE algorithm to work with it. 

The first test consists on simply doing a random feature selection and plotting the accuracy for the training and test splits (Figure \ref{fig:dynamicStep.madelon.base}). This random selection algorithm will allow us to make comparisons with other feature selection algorithms to determine whether they work or not. Because we're using the average of many random selection experiments, the curve it produces should be always worse than that of any other working selection algorithm. It is expected that it will produce a somewhat linear function with accuracy dropping consistently when the amount of features selected is reduced.

\begin{figure}[H]
    \centering
    \begin{subfigure}[b]{0.4\linewidth}
        \includegraphics[width=\linewidth]{img/ch5/madelon-random-c_1}
        \subcaption*{$C=1$}
    \end{subfigure}
    \begin{subfigure}[b]{0.4\linewidth}
        \includegraphics[width=\linewidth]{img/ch5/madelon-random-c_1}
        \subcaption*{$C=10^3$}
    \end{subfigure}
    \caption{Accuracy of an SVM classifier with a random feature selection with the Madelon dataset.}
    \label{fig:dynamicStep.madelon.base}
\end{figure}

The results of this test are a bit unexpected. Although the training accuracy behaves as expected, the test accuracy is way too low for even the case where all features are used (0.55). Reducing the amount of features, even randomly, produces a slight increase in the test accuracy. This is a fine example of the effects of the curse of dimensionality, on how even a random selection, which should produce consistently worse models the less amount of features it has, ends up producing better ones simply because there are fewer features with independence on how the feature selection is made.

We've performed a second test to see if the regularization parameter has any effect and can be of any help in trying to reduce overfitting (Figure \ref{fig:dynamicStep.madelon.reg}).

\begin{figure}[h]
    \centering
    \includegraphics[width=0.4\linewidth]{img/ch5/madelon-cv-c}
    \caption{Test accuracy mean of SVM classifier with 10-fold cross-validation using all features for multiple values of C.}
    \label{fig:dynamicStep.madelon.reg}
\end{figure}

As the plot indicates, no value of the regularization parameter helps improve the performance. These poor results suggest that the dataset is not linearly separable and thus is producing huge amounts of overfitting in all cases, but, how does this affect SVM-RFE?

To see how SVM-RFE performs under these conditions we've designed an ex\-periment very similar to that used in random selection, although instead of using 10 random selections we've used cross-validation to take the mean. In this experiment we also use a constant step of 10 for performance reasons, see Figure \ref{fig:dynamicStep.madelon.step10}.

\begin{figure}[h]
    \centering
    \includegraphics[width=0.4\linewidth]{img/ch5/madelon-svmrfe-step-s_10}
    \caption{Mean accuracy using 20-fold CV of an SVM classifier with SVM-RFE selection of the Madelon dataset.}
    \label{fig:dynamicStep.madelon.step10}
\end{figure}

Comparing this plot with that of the random selection we can see that the results are quite disappointing. Although SVM-RFE does improve the performance on the training data, the test performance remains the same. With this we can conclude that SVM-RFE using a lineal kernel does not work well on this specific dataset, and thus, no matter the step strategy used, the results will all be quite bad. 

We have also checked how does SVM-RFE perform if we use a dynamic step strategy (Figure \ref{fig:dynamicStep.madelon.dstep}), but no significant difference is appreciated.

\begin{figure}[H]
    \centering
    \begin{subfigure}[b]{0.4\linewidth}
        \includegraphics[width=\linewidth]{img/ch5/madelon-svmrfe-dstep-p_01.png}
        \subcaption*{10\%, 52 iterations, stop at 0}
    \end{subfigure}
    \begin{subfigure}[b]{0.4\linewidth}
        \includegraphics[width=\linewidth]{img/ch5/madelon-svmrfe-dstep-p_003.png}
        \subcaption*{3\%, 136 iterations, stop at 50}
    \end{subfigure}
    \caption{Mean accuracy using 20-fold CV of an SVM classifier with SVM-RFE selection using dynamic step of the Madelon dataset.}
    \label{fig:dynamicStep.madelon.dstep}
\end{figure}


This is a problem. We've initially wanted to use the Madelon dataset to do our experiments, but these results show that the dataset in not adequate for a linear classifier. Thus, we've decided to use simpler artificially generated datasets for this experiment and come back to the Madelon dataset later on with other extensions of the SVM-RFE algorithm.

\subsubsection*{Initial Analysis with artificially generated data}

To generate the dataset we've used the \textit{make\_classification} function of the \texttt{Sklearn} library. We've created a dataset with 300 features, 50 of which are informative. Ideally, a perfect feature selection algorithm should be able to produce a peak test accuracy at the exact amount of informative features or less.

Similar to the experiments descried before, we've performed random selection and SVM-RFE. Here tough, we also make a comparison with different values of the step to see how it effects the actual performance of the selection. In the particular case where the step is equal to the amount of features this is no longer SVM-RFE but a filter method based on SVM (one single iteration).

\begin{figure}[H]
    \centering
    \begin{subfigure}[b]{0.4\linewidth}
        \includegraphics[width=\linewidth]{img/ch5/vanilla300-random.png}
        \subcaption*{\textbf{A.} Random selection}
    \end{subfigure}
    \begin{subfigure}[b]{0.4\linewidth}
        \includegraphics[width=\linewidth]{img/ch5/vanilla300-svmrfe-s300.png}
        \subcaption*{\textbf{B.} SVM-based filter method, step = 300}
    \end{subfigure}
    \begin{subfigure}[b]{0.4\linewidth}
        \includegraphics[width=\linewidth]{img/ch5/vanilla300-svmrfe-s10.png}
        \subcaption*{\textbf{C.} SVM-RFE, step = 10}
    \end{subfigure}
    \begin{subfigure}[b]{0.4\linewidth}
        \includegraphics[width=\linewidth]{img/ch5/vanilla300-svmrfe.png}
        \subcaption*{\textbf{D.} SVM-RFE, step = 1}
    \end{subfigure}
    \caption{Mean accuracy using 20-fold CV of an SVM classifier with different selection methods.}
    \label{fig:dynamicStep.vanilla.comp}
\end{figure}

From this analysis we conclude that SVM-RFE works well on this dataset. We also verify, by comparing plots \textbf{B} and \textbf{C}, our previous assumption that the more iterations (smaller step) the better the accuracy. This relationship, however, doesn't seem to be lineal, as the same effect can not be appreciated when comparing plots \textbf{C} and \textbf{D}.

Notice that when evaluating the performance of a feature selection algorithm two variables must be considered, one is accuracy performance and the other is amount of features. A selection with a great accuracy but a lot of features may not be as good as a section with a decent accuracy and only a handful of features. That is, we want to maximize accuracy while minimizing amount of features. These two variables do not operate on the same range, and it can even be the case that the importance of one over the other is problem-dependent. This makes comparing multiple instances hard, this is why are comparing pairs of plots instead of simply a pair of values.

\subsubsection*{Dynamic Step vs Constant Step}

Based on our assumptions, dynamic step can never beat in accuracy performance a constant step of one. Beating constant steps bigger than 1 should theoretically be possible when the amount of features is huge, since at some point the specific step used will be smaller than the constant, and there an improvement may be found. For our specific dataset of 300 features however, no such improvement was found, likely because the amount of features was insufficient, but also because of the no linearity of the stated property.

Rather than for improving accuracy, dynamic step is more useful for improving computational cost. Because each iteration reduces the amount of features, training the SVM will be faster with dynamic step even if the same amount of iterations are used, since there will be more iterations using fewer features in the dynamic step case.

This same mechanic does not apply only to using percentages, the other methods stated at section \ref{sec:dynamicStep.desc} follow the same logic. In Figure \ref{fig:dynamicStep.compare} you can see a comparison of dynamic step methods. Although all of them draw some kind of curve, they don't use the same shape, and thus can produce slightly different performance results using the same amount of iterations. The time cost of the methods can be estimated as the area under the curve.

\begin{figure}[h]
    \centering
    \includegraphics[width=0.4\linewidth]{img/ch5/comparetimes.png}
    \caption{Amount of features remaining (vertical axis) at each iteration (horizontal axis).}
    \label{fig:dynamicStep.compare}
\end{figure}

Experimentally we confirm our results and see that using a percentage is faster compared to a constant step, without reducing its accuracy. The following table summarizes the time cost for various experiments. The SVM-RFE time shown is the mean of the times of every fold used during cross-validation.

\begin{table}[h]
    \centering
    \begin{tabular}{l r r r}
    \toprule
    \tabhead{Strategy} & \tabhead{Iterations} & \tabhead{Time} & \tabhead{Total Time} \\
    \midrule
    Constant & 60 & 29.16s & 00:02:36\\
    Constant & 30 & 15.90s & 00:02:00\\
    Constant &  6 & 4.43s  & 00:01:29\\
    Constant &  3 & 2.74s  & 00:01:22\\
    Percentage & 47 & 7.03s & 00:01:29\\
    Percentage & 26 & 3.83s & 00:01:17\\
    Percentage & 10 & 2.59s & 00:01:13\\
    \bottomrule\\
    \end{tabular}
    \caption{Time costs for various experiments.}
    \label{tab:dynamicStep.times}
\end{table}

Notice that although one could think that the smaller the area under the curve the better, and it is true when it comes to time costs, a very pronounced curve would imply a very big initial step, this can result in a drop in accuracy performance in the very first iterations that can not be restored later on even with a step of 1. This is specially important for the annealing method, since it chooses a 50\% percentage in the very first iteration. How much acute the curve can be without producing a drop in accuracy depends on the dataset, specifically on the amount of informative features. Thus, introducing a hyperparameter to specify the sharpness of the curve may be useful. The percentage approx already has this hyperparameter by default, but introducing such parameter in the other methods is also trivial. Another option would be to further customize the curve by introducing minimum and maximum steps.

Finally, one last parameter that may be of relevance is the stop value. If the amount of informative variables is known, it doesn't make sense to perform many iterations on selections so small that the performance has already dropped, instead it may make more sense to increase the amount of iterations the closer you get to the amount of informative variables. This effect can be seen in Figure 

\begin{figure}[H]
    \centering
    \begin{subfigure}[b]{0.4\linewidth}
        \includegraphics[width=\linewidth]{img/ch5/vanilla300-svmrfe-stopA.png}
        \subcaption*{2\%, 26 iterations, stop at 0}
    \end{subfigure}
    \begin{subfigure}[b]{0.4\linewidth}
        \includegraphics[width=\linewidth]{img/ch5/vanilla300-svmrfe-stopB.png}
        \subcaption*{2\%, 25 iterations, stop at 20}
    \end{subfigure}
    \caption{Comparison of the dynamic step behavior with and without using a stop parameter.}
    \label{fig:dynamicStep.vanilla.stop}
\end{figure}

\subsection{Annotations}

\begin{itemize}
    \item The Cross Validation procedure has been implemented by hand and par\-al\-lelized.
    \item How do I know which method is better?
    \item How do I measure how good some method is (accuracy vs selection size)? Mathematically.
    \item Should I do hyperparameter search (of C) also here?
    \item Explain why plotting takes so much time.
    \item Should the percentage really be a hyperparameter? Annealing and Square Root do not, and it doesn't make much of a difference anyways?
    \item The complexity of SVM seems to be $O(\max(n, d) \min(n, d)^2)$. \\ If dimensions > nº observations: $O(dn^2)$. \\ Otherwise: $O(nd^2)$.
\end{itemize}

% -----------------------------------------------------------------------------------------

\section{Sampling}

\section{Stop Condition}

% -----------------------------------------------------------------------------------------

\section{Multi-Class}

In this section we extend SVM-RFE to the multi-class classification problem.

\subsection{Description and reasoning}
\label{sec:stopCond.desc}

When it comes to extending SVM to handle a multi-class problems two common methods exists, OvR (One-vs-Rest) and OvO (One-vs-One). In both cases the idea is to divide the problem in a set of binary classification problems and use a joint decision function that operates on the results of each of these. Because we're not really making any predictions during the SVM-RFE procedure, we can not use this joint decision function. Instead, we must find a way to merge the ranking criteria obtained form each problem to find a joint ranking criteria.

We know that the ranking criteria is an estimator of the importance of some feature for a given binary decision problem. It can be the case that a feature is very useful to distinguish between two classes but useless for the rest. In this case a joint ranking criteria formed by taking the mean, the median or the sum will result in poor selections. A better idea would be to take the maximum. However, it may also be desirable to estimate the joint importance a feature has by considering its individual importance in more than one problem, that is, a feature that is important in more than one binary classification problem is more important than another that is only important in one such classification problem even if the second feature has a greater individual importance. A way to perform such ranking would be, for instance, the sum of the squares. For both methods proposed a normalization of all feature rankings is probably adequate.

Note that \texttt{sklearn} only supports OvO, and is therefore the option we will use.

\subsection{Pseudocode formalization}

\textbf{Definitions:}

\begin{itemize}
    \item $X_0 = [\vt{x_0}, \vt{x_1}, \dotsc, \vt{x_k}]^T$ list of observations.
    \item $\vt{y} = [y_1, y_2, \dotsc, y_k]^T$ list of labels.
\end{itemize}

\begin{algorithm}[H]
    \DontPrintSemicolon
      \KwInput{$t$ \tcp*{$t$ = step}}
      \KwOutput{$\vt{r}$}
      \KwData{$X_0,\vt{y}$}
      $\vt{s} = [1,2, \dotsc, n]$ \tcp*{subset of surviving features}
      $\vt{r} = []$ \tcp*{feature ranked list}
      \While{$|\vt{s}| > 0$}
        {
            \tcc*[h]{Restrict training examples to good feature indices}\\
            $X=X_0(:,\vt{s})$\VS

            \tcc*[h]{Compute the joint ranking criteria}\\
            $\vt{c} = [0, 0, \dots, ]$\\
            \For{$\vt{Xl} \subseteq \vt{X}$, $\vt{yl} \subseteq \vt{y}$ with $\vt{Xl}$ and $\vt{yl}$ being an instance of OvO}{
                \tcc*[h]{Train the classifier}\\
                $\vt{\alpha} = \texttt{SVM-train(} \vt{Xl}, \vt{yl} \texttt{)}$\VS

                \tcc*[h]{Compute the weight vector of dimension length $|\vt{s}|$}\\
                $\vt{w} = \sum_k{\vt{\alpha_k} \vt{yl_k} \vt{Xl_k}}$\VS
    
                \tcc*[h]{Compute the joint ranking criteria}\\
                $\vt{c} = [\max(c_i, (w_i)^2) \text{ for all $i$}]$\VS
            }\VS

            \tcc*[h]{Find the $t$ features with the smallest ranking criterion}\\
            $\vt{f} = \texttt{argsort}(\vt{c})(\ :t)$\VS

            \tcc*[h]{Iterate over the feature subset}\\
            \For{$f_i \in \vt{f}$}{
                \tcc*[h]{Update the feature ranking list}\\
                $\vt{r} = [\vt{s}(f_i), ...\vt{r}]$\VS
    
                \tcc*[h]{Eliminate the feature selected}\\
                $\vt{s} = [...\vt{s}(1:f_i - 1), ...\vt{s}(f_i + 1:|\vt{s}|)]$
            }
        }
    \caption{SVM-RFE for multi-class classification problems}
    \label{alg:svmrfe-stopcond}
\end{algorithm}

\subsection{Results}

\section{Non-linear Kernels}

\section{Combo}


----

In this section 

\begin{itemize}
    \item Importance of scaling preprocessing
    \item Complexity analysis and pseudocode.
\end{itemize}
% Chapter 1

\chapter{Conclusions} % Main chapter title

\section{Problems encountered during development}

\subsection*{Changes to the planning}

Although the writing of the theory section of this thesis was planned for the final phase, it has been switched to the first phase together with research. This has was needed because, in order to have a clear understanding of the ongoing research, it was useful to write it down. Having a written version of the research helps clarify ideas and have an immediate source of truth for the problems encountered while designing the experiments.

This was specially important for the theory on kernels of SVM and the theory of ranking criteria of SVM-RFE. Without a clear understanding of it, the non-linear kernel extension (a cornerstone experiment) would not been possible to implement.

Some delays also occurred due to unforeseen circumstances. These forced some meetups to be pushed to a later date and the development to stop partially or totally. The specific circumstances where:

\begin{itemize}
    \item 2 week delay caused by medical problems.
    \item 1 week delay caused by ransomware infection on a managed (family business) server. 
\end{itemize}

Since we had already made the budget flexible, no significant changes were required.

\subsection*{Cancelled extensions}

\subsubsection*{Avoid CV}
Our definition consisted in using other classifier algorithms, such as LDA (Linear Discriminant Analysis) or Logistic Regression. These however fall too much outside the scope of this project. If they were to be introduced, the theory for both would also need to be written. The theory about the specific characteristics that may make them more suitable, or not, for the RFE procedure would need to be researched as well.

Not only a lot of work would have to be put on researching classifiers that do not really have a direct connection with the SVM-RFE algorithm itself, but this work would also be incompatible with non-linear kernels (due not direct application of the general ranking criteria formula), thus creating two disjoint research.

\subsubsection*{Historic of weight}
This was dropped soon after realizing how the ranking criteria of SVM-RFE works. It is stated in the theory that weights of previous iterations will necessarily be worse estimators of feature importance compared to the current iteration. Therefore, doing a mean, or a similar average, will inevitably lead to worse performance.

Two alternatives have been proposed but also dropped:

\begin{enumerate}
    \item Utilize the variance accumulated in all previous iterations in some sensible manner.
    \item Instead of using the weight, use the alpha values of the last iteration to init\-ialize the SVM optimization problem. This is expected to reduce the comput\-ational cost required to reach a solution.
\end{enumerate}

The first alternative lacks any kind of theoretical background, and is likely to not perform any better. Is thus, simply not attractive enough of an experiment to pursue (high chance of failing to produce any improvement at all).

The second alternative is more interesting but has one major flaw. In our current framework we're not implementing the quadratic solver program required to solve SVM, instead we're using the C++ implementation LIBSVM/LIBLINEAR. We're not using C++, we're using Python. This means that in order to use this implementation we're limited to a Python API that performs the binding. This API does not expose the alpha values on initialization, only as a final result. In order to make these aviable we would need to make changes to the C++ implementation, recompile, bind, and then modify the Python API. This is way too much outside the scope of this project and will take much more time than that allocated for correcting bugs in the initial plan.

\subsubsection*{Stop condition}

Initially this extension was planed as a standalone method to reduce the cost in time of the RFE algorithm. Although the experiment itself was a success, that is, the implementation worked as expected, the resulting gain in performance was minimal due to the fact that the stop point (optional feature subset size) is often in the last iterations, and these are precisely the ones that are less computationally expensive.

We reformulated the experiment as method whose objective was merely to find an approximation to of the optional feature subset size, and then use that in\-for\-ma\-tion to improve on the dynamic step extension.

\subsection*{Non-linear Kernel}

This has clearly become the bulk of the project as it required the most effort for both the research and the implementation. Various problems where found during the implementation phase:

\subsubsection*{No API for retrieving the Kernel Matrix}
\texttt{Sklearn} does not provide an API that allows retrieving the kernel matrix. Usually it computes the kernel internally based on its name and parameters, and hides it from the user. Fortunately, \texttt{sklearn} does provide a precomputed mode that allows computing the kernel matrix yourself and pass it to the solver. This mode is poorly documented, and it was necessary to read to source code to find how to handle it.

Using this mode also slightly increased the computational cost of SVM-RFE, even though \texttt{sklearn} own functions where being used to generate the kernel matrix.

\subsubsection*{Slow optimizations}
Experimentally we found that computing the ranking criteria took about 90\% of the time on each run of SVM-RFE with the naive implement\-ation. This is grounds for applying the known (but not formalized) optimizations mentioned on the original research paper. This consists on caching results from previous computations and restricting the comput\-ation to support vectors.

To apply these optimizations we had to drop the \texttt{sklearn} kernel function im\-ple\-men\-ta\-tion and made our own. This proved immediately problematic because we could not use \texttt{numpy} for most computations and had to rely on explicitly declaring a double loop. \texttt{Numpy} speeds up computations by relying on a low level compiled implementation of its functions. By not using it, our equivalent implementation in pure Python performed much slower (in the order of 70 times slower).

We mitigated this problem by using a library called \texttt{numba}. This library compiles selected Python functions so that we can get similar speeds to what we would get with an implementation made in a compiled language. Still, our implementation in \texttt{numba} was about 3 times slower than with \texttt{sklearn}.

Applying the optimizations we managed to get a speed increase by a factor of 3. Although the code is slower, the complexity is one degree faster, from $O(dn^2)$ for the \texttt{sklearn} version to $O(n^2)$ for ours.

\section{Closing thoughts}

We believe to have found some useful techniques to improve the computational cost of SVM-RFE, as shown in the \emph{Combo} extension. Each of these techniques, however, has their own limits. It remains to be seen how useful these techniques would be in practice as an off-the-shelf method. For this, more extensive testing needs to be done.

We've implemented SVM-RFE for non-linear kernels, and some optimizations that where not widely known. This may prove useful to future developers.


%----------------------------------------------------------------------------------------
%	BIBLIOGRAPHY
%----------------------------------------------------------------------------------------

\printbibliography[heading=bibintoc]

%----------------------------------------------------------------------------------------

\end{document}  
